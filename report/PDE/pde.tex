\section{Biharmonic equation}
\label{sec:PDE}

\newtheorem{theorem}{Theorem}
\newtheorem{lemma}{Lemma}
\theoremstyle{remark}
\newtheorem*{remark}{Remark}

\theoremstyle{definition}
\newtheorem{definition}{Definition}

Consider the inhomogeneous Biharmonic equation with clamped boundary conditions on the unit square $\Omega = [0, 1]^2$:
\begin{subequations}\label{eq:PDE}
  \begin{equation}
    \nabla^4 u = f \quad, (x, y) \in \Omega,
  \end{equation}
  \begin{equation}
    u = 0, \nabla^2u = 0 \quad, (x, y) \in \partial\Omega.
  \end{equation}
\end{subequations}

\subsection{Analytical solution}

\begin{figure}
\centering
\pgfplotsset{colormap={redblue}{rgb=(0,0,1) rgb=(1,1,1) rgb=(1,0,0)}}
\begin{tikzpicture}
\begin{axis}[
	width=7.5cm, height=7.5cm,
	xtick={-1,0,1}, ytick={-1,0,1},
	xlabel=$x$, ylabel=$y$,
	title={$u(x,y)$},
	% grid=major,
	declare function={
		myfunc(\x,\y)=sin(deg(pi*\x))*sin(deg(pi*\y)); % + 0.5*sin(deg(3*pi*\x))*sin(deg(3*pi*\y));
	},
	colormap name=redblue,
	view={0}{90},
	% colorbar right, colorbar style={ytick={-1,0,+1}},
]
\addplot3 [surf, domain=-1:1, contour filled={number=50}] {myfunc(x,y)};
\draw (+0.5, +0.5) node {$u(x,y)$};
\draw (-0.5, +0.5) node {$-u(x+1,y)$};
\draw (+0.5, -0.5) node {$-u(x,y+1)$};
\draw (-0.5, -0.5) node {$u(x+1,y+1)$};
\draw [->, thick] (+0.1, +0.5) -- (-0.1, +0.5);
\draw [->, thick] (+0.5, +0.1) -- (+0.5, -0.1);
\draw [->, thick] (-0.5, +0.1) -- (-0.5, -0.1);
\draw [->, thick] (+0.1, -0.5) -- (-0.1, -0.5);
\end{axis}
\end{tikzpicture}
\hfill
\begin{tikzpicture}
\begin{axis}[
	width=8cm, height=8cm,
	xtick={-1,0,1}, ytick={-1,0,1},
	xlabel=$x$, ylabel=$y$, zlabel={$u(x,y)$},
	% grid=major,
	declare function={
		myfunc(\x,\y)=sin(deg(pi*\x))*sin(deg(pi*\y)); % + 0.5*sin(deg(3*pi*\x))*sin(deg(3*pi*\y));
	},
	colormap name=redblue,
	% view={0}{90},
	% colorbar right, colorbar style={ytick={-1,0,+1}},
]
\addplot3 [surf, domain=-1:1] {myfunc(x,y)};
\end{axis}
\end{tikzpicture}
\caption{\label{continuationfig}
	The function $u(x,y)$ is originally defined on $[0,1]\times[0,1]$, but we extend the definition to the full $xy$-plane with the rules $u(x+1,y) = u(x,y+1) = -u(x,y)$.
	This makes $u(x,y)$ periodic and permits Fourier analysis.
	Here is the continuation on $[-1,1]\times[-1,1]$.
}
\end{figure}

To use Fourier analysis, let us extend the definition of $u(x,y)$ on $[0,1]\times[0,1]$ to the full $xy$-plane $[-\infty,+\infty]\times[-\infty,+\infty]$ as the antisymmetric continuation with the rules $u(x,y+1) = u(x+1,y) = -u(x,y)$.
The procedure is illustrated in \cref{continuationfig}.
Using the antisymmetry, the conditions $u = -u = 0$ and $\nabla^2 u = -\nabla^2 u = 0$ are automatically satisfied at the boundaries.
This can also be seen for the simple trial function in \cref{continuationfig}, which vanishes and inflects at the boundaries.
Now $u(x,y) = u(x+2,y) = u(x,y+2)$ is periodic in both directions with period $2$ and can therefore be written as a Fourier series
\begin{equation*}
	u(x,y) = \sum_{m=-\infty}^{+\infty} \sum_{n=-\infty}^{+\infty} \hat{u}_{mn} e^{i 2 \pi m x / 2} e^{i 2 \pi n y / 2}
\end{equation*}
with coefficients (TODO: note on orthogonality or take this as known?)
\begin{equation*}
	\hat{u}_{mn} = \frac{1}{4} \integraltwo{u(x,y) e^{-i 2 \pi m x / 2} e^{-i 2 \pi n y / 2}}{x}{y}{-1}{+1}{-1}{+1}.
\end{equation*}
By the antisymmetry $u(x+1,y)=u(x,y+1)=-u(x,y)$ and the symmetry $u(x+1,y+1)=u(x,y)$, the Fourier coefficients satisfy
\begin{equation*}
	u_{m,n} = -u_{-m,n} = -u_{m,-n} = u_{-m,-n},
\end{equation*}
so $\hat{u}_{00} = -\hat{u}_{00} = 0$ and we can write the Fourier series as
\begin{equation*}
\begin{split}
	u(x,y) &= \sum_{m=1}^{+\infty} \sum_{n=1}^{+\infty} \hat{u}_{mn}
	\left( 
	  e^{+i \pi m x} e^{+i \pi n y}
	- e^{-i \pi m x} e^{+i \pi n y}
	- e^{+i \pi m x} e^{-i \pi n y}
	+ e^{-i \pi m x} e^{-i \pi n y}
	\right) \\
	       &= -4 \sum_{m=1}^{+\infty} \sum_{n=1}^{+\infty} \hat{u}_{mn} \sin(m \pi x) \sin(n \pi y)
\end{split}
\end{equation*}
after simplifying all complex expontentials using Euler's identity $e^{ix} = \cos x + i \sin x$.
Rescaling $\hat{u}_{mn} \rightarrow -\hat{u}_{mn}/4$, we then begin by expressing our analytical solution as the double sine series
\newcommand{\fourierseries}[3]{
	\sum_{#2=1}^{\infty} \sum_{#3=1}^{\infty} #1 \sin(#2 \pi x) \sin(#3 \pi y)
}
\newcommand{\fourierexpand}[1]{
	\fourierseries{\hat{#1}_{mn}}{m}{n}
}
\begin{equation}
u(x,y) = \fourierexpand{u}.
\label{pde:equation:fourierexpansion}
\end{equation}

Plug this Fourier series into \cref{eq:PDE} and act with the biharmonic operator to get
\begin{equation*}
\nabla^4 u(x,y) = \fourierseries{\left((m\pi)^2+(n\pi)^2\right)^2 \hat{u}_{mn}}{m}{n} = f(x,y).
\end{equation*}
For simplicity, we \textbf{restrict ourselves to sources that can also be written} (TODO: what restrictions does this impose on $f$?)
\begin{equation}
	f(x,y) = \fourierexpand{f}.
\end{equation}

The Fourier series for $\nabla^4 u(x,y)$ and $f(x,y)$ can be equal only if their coefficients are equal.
This can be seen formally by multiplying both by $\sin(2m'\pi x) \sin(2n'\pi y)$, integrating over $x$ and $y$ and using orthogonality of the sine functions,
\begin{equation*}
	\integral{\sin(m\pi x) \sin(m' \pi x)}{x}{0}{1} = \frac{1}{2} \delta_{mm'}.
\end{equation*}
Therefore, the coefficients of the solution are
\begin{equation}
	\hat{u}_{mn} = \frac{\hat{f}_{mn}}{\left((m\pi)^2+(n\pi)^2\right)^2},
\end{equation}
and the solution $u(x,y)$ is available by summing its Fourier series \ref{pde:equation:fourierexpansion}.

If we know $\hat{f}_{mn}$, it is straightforward to compute $\hat{u}_{mn}$ and thus the solution $u(x,y)$ itself from its Fourier series.
If we only know $f(x,y)$, we can find the coefficients by using the orthogonality of the sine functions again.
Multiply the Fourier series by $\sin(m' \pi x) \sin(n' \pi y)$ and integrate over $x$ and $y$ to get
\begin{equation*}
\begin{split}
  & \,\, \integraltwo{f(x,y) \sin(m' \pi x) \sin(n' \pi y)}{x}{y}{0}{1}{0}{1} \\
= & \,\, \sum_{m=1}^{\infty} \sum_{n=1}^{\infty} \hat{f}_{mn} \underbrace{\integral{\sin(m\pi x)\sin(m'\pi x)}{x}{0}{1}}_{\delta_{mm'}/2} \underbrace{\integral{\sin(\pi y)\sin(n'\pi y)}{y}{0}{1}}_{\delta_{nn'}/2} \\
= & \,\, \hat{f}_{m'n'} / 4.
\end{split}
\end{equation*}
Read from bottom to top,
\begin{equation}
\hat{f}_{mn} = 4 \integraltwo{f(x,y) \sin(m \pi x) \sin(n \pi y)}{x}{y}{0}{1}{0}{1}.
\end{equation}

Note that a general source $f(x,y)$ may only be represented exactly by an infinite Fourier series.
To make the Fourier series solution viable, we must cut it off to include only a finite number of terms.
In this case, we should analyze $f(x,y)$ to make sure that we exclude only Fourier modes that contribute insignificantly to the solution.
However, we can also construct problems with a finite number of terms in the \emph{exact} solution by simply defining the source $f(x,y)$ in terms of a finite number of nonzero Fourier coefficients.

%% b)

\subsection{Numerical solution method}

We may transform \eqref{eq:PDE} into a system of Poisson equation by introducing $g = \nabla^2 u$:
\begin{align}\label{eq:PDE-poisson}
  \nabla^2g &= f,\\
  \nabla^2u &= g,\\
  g = u &= 0,\quad \text{ on } \partial \Omega.
\end{align}


%% c)
In solving the poisson equation $\nabla^2 u = f$ we will consider two discretization schemes, the five point stencil and nine point stencil, which we will denote $\nabla_5^2$ and $\nabla_9^2$.

\newcommand{\crossStencil}[5]{%
  \begin{tikzpicture}[scale=0.5,baseline=1mm, every node/.style={scale=0.7}, nodes={draw, circle}]
    \draw node[below]{$#1$} (0,0) -- (0,2) node[above]{$#2$};
    \draw (-1, 1) node[left]{$#3$} -- (1, 1) node[right]{$#4$};
    \node[above right, draw=none] at (0,1) {$#5$};
  \end{tikzpicture}
}

\newcommand{\xStencil}[4]{%
  \begin{tikzpicture}[scale=0.5,baseline=1mm, every node/.style={scale=0.7}, nodes={draw, circle}]
    \draw node[below left]{$#1$} (0,0) -- (1.41, 1.41) node[above right]{$#2$};
    \draw (0, 1.41) node[above left]{$#3$} -- (1.41, 0) node[below right]{$#4$};
  \end{tikzpicture}
}

Written in stencil diagrams, the five point stencil is given as \cite{part2}
$$
u \left(\crossStencil{1}{1}{1}{1}{-4}\right) = h^2 f
$$
while the nine point stencil is given by
$$
u
\left(
\xStencil{\frac16}{\frac16}{\frac16}{\frac16}
+
\crossStencil{\frac23}{\frac23}{\frac23}{\frac23}{-\frac{10}{3}}
\right)
=
h^2
f
\left(
\crossStencil{\frac1{12}}{\frac1{12}}{\frac1{12}}{\frac1{12}}{\frac23}
\right).
$$


We will now describe the structure of the matricies involved in the problem, starting with the simpler five point stencil.
We will use the same flattening for the 2D discrete function $u$ as described in section \ref{sec:exc3:numerical}.
We recognize that in $U$ neighbouring elements correspond to grid points that are left and right of each other, while the grid point up and down is the element N places before and after.
Thus, writing out the five point stencil $K2D$, we get a toeplitz and symmetric matrix, where the main diagonal has -4 and the $\pm 1$ and $\pm N$ off diagonals have 1.
We can compactley write this as
$$
K2D =
\begin{bmatrix}
  K & 0 \\
  0 & K & 0\\
  0 & 0 & K \\
  &&&\ddots
\end{bmatrix}
+
\begin{bmatrix}
  -2I & I & 0 &  \\
  I & -2I & I & 0 \\
  0 & I & -2I & I & 0 \dots\\
  \vdots&&&\ddots
\end{bmatrix}.
$$
where $I$ is the identity matrix and $K$ is the one dimensional central finite difference matrix of order 2,
$$
K =
\begin{bmatrix}
  -2 & 1 &   \\
  1 & -2 & 1 &  \\
  & 1 & -2 & 1 & \\
  &&&\ddots\\
  && 1 & -2 & 1\\
  &&& 1 & -2
\end{bmatrix}.
$$
Using the Kronecker product, this may be written as
$$
% K2D = \verb|kron|(I, K) + \verb|kron|(K, I)
K2D = K \otimes I + I \otimes K = K \oplus K,
$$
where $\otimes$ and $\oplus$ are the Kronecker product and Kronecker sum.
This may also be useful when working with sparse matricies, as efficient methods for the Kronecker product are implemented in frameworks such as Scipy\cite{scipy_kron}.

We will split the nine point stencil matrix into two, according to the stencil diagram above.
One matrix represents a five point stencil with weights altered to $\frac{-10}{3}, \frac23$ instead of $-4, 1$, we denote this matrix by $K2D^{(9)}$.
The other is the ``X''-stencil taking care of the NE, NW, SE, and SW points, each with a weight of $\frac16$, as shown in the diagram.
We will denote this matrix by $\Sigma 2D$.

Let
\begin{equation}
  \Sigma =
  \frac{1}{\sqrt{6}}
  \begin{bmatrix}
    0 & 1  \\
    1 & 0 & 1 \\
      & 1 & 0 & 1 \\
      &   & 1 & 0 & \ddots\\
      &   &   & \ddots  & \ddots
  \end{bmatrix}
\end{equation}
be the TST matrix with ones on its off-diagonals and zero on the diagonal.
Then we have $\Sigma 2D = \Sigma \otimes \Sigma$.
The nine point stencil is then
\begin{equation}
  somename9stencil = K2D^{(9)} + \Sigma 2D
\end{equation}

Let $\lambda_k, ~k\in[1, n]$ be the eigenvalues of the matrix $A$, and let $\mu_l, ~l\in [1, m]$ be the eigenvalues of the matrix $B$.
As shown in for example \cite{Laub_2004} the eigenvalues of the Kronecker product $A \otimes B$ are the products of the eigenvalues of $A$ and $B$,
$$
%% \lambda_1 \mu_1, \dots, \lambda_1 \mu_m,
%% \lambda_2 \mu_1, \dots, \lambda_2 \mu_m,
%% \dots,
%% \lambda_n \mu_1, \dots, \lambda_n \mu_m.
\lambda_k\lambda_l, \quad k \in [1, n], l\in [1,m].
$$
The eigenvalues of the Kronecker sum $A\oplus B$ are the sum of the eigenvalues of $A$ and $B$
$$
%% \lambda_1 + \mu_1, \dots, \lambda_1 + \mu_m,
%% \lambda_2 + \mu_1, \dots, \lambda_2 + \mu_m,
%% \dots,
%% \lambda_n + \mu_1, \dots, \lambda_n + \mu_m.
\lambda_k + \lambda_l, \quad k \in [1, n], l\in [1,m].
$$
%(TODO: these are non-obvious properties, need a reference, e.g. \url{http://www.siam.org/books/textbooks/OT91sample.pdf} theorem 13.12 and 13.16)
% Some properties/proofs in https://www.uio.no/studier/emner/matnat/ifi/nedlagte-emner/INF-MAT3350/h07/undervisningsmateriale/chap9slides.pdf
In general the eigenvalues of a TST matrix are\cite{tridiagonal}
\begin{equation*}
	\lambda_k = a + 2 b \cos\left(\frac{k \pi}{N + 1}\right)
\end{equation*}
where $a$ is the main diagonal and $b$ is the off diagonal.
%% Thus the eigenvalues of $K$ are
%% $$
%% \lambda_k = -2
%% \left(1 - \cos\left(\frac{k \pi}{N + 1}\right)\right)
%% $$
%% and the eigevalues of $\Sigma$ are
%% $$
%% \gamma_k = -\frac{2}{\sqrt{6}}\cos\left(\frac{k \pi}{N + 1}\right).
%% $$
Thus, the eigenvalues of $K2D$ are
$$
\left(-2 + 2\cos\left(\frac{k \pi}{N + 1}\right)\right)
+
\left(-2 + 2\cos\left(\frac{l \pi}{N + 1}\right)\right).
$$
The eigenvalues of $K2D^{(9)}$ are
$$
\frac13\left(-10 + 4\cos\left(\frac{k \pi}{N + 1}\right)\right)
+
\frac13\left(-10 + 4\cos\left(\frac{l \pi}{N + 1}\right)\right)
$$
and the eigenvalues of $\Sigma \otimes \Sigma$ are
$$
\frac46
\cos\left(\frac{k \pi}{N+1}\right)
\cos\left(\frac{l \pi}{N+1}\right).
$$
The eigenvalues of the nine point stencil are thus
\begin{equation}
  -\frac{10}{3}
  + \frac43
  \left(
  \cos(\frac{k \pi}{N+1})
  + \cos(\frac{l \pi}{N+1})
  \right)
  +
  \frac46
  \cos(\frac{k \pi}{N+1})
  \cos(\frac{l \pi}{N+1}).
\end{equation}

%% Let us defne an infinity norm for convenience
\newcommand{\inorm}[1]{
\lVert #1 \rVert_\infty
}

\subsection{Stability and order of the five and nine point stencils}
\begin{definition}
  The $n$-point stencil $\nabla_n^2 f(x_0) = \sum_{i=0}^{n-1} a_i f(x_i)$,
  where $x_i$ are the neighbouring points of $x_0$.
\end{definition}

\begin{definition}
  We denote by a proper stencil a stencil where $a_i > 0$ for $i>0$ and $\sum_{i=1}^n a_i = -a_0$.
\end{definition}

\begin{lemma}\label{pde:lemma:max}
If $\nabla_n^2 f \geq 0$ on $\Omega$, and $\nabla_n^2$ is a proper stencil, the maximum value of $f$ is attained on $\partial \Omega$, that is
$$
\max(f)_\Omega \leq \max(f)_{\partial \Omega}.
$$
\end{lemma}
\begin{proof}
Suppose the opposite is true,
$$
\max(f)_\Omega > \max(f)_{\partial \Omega}.
$$
Then, there is an internal grid point $x_0$ on which $f$ attains it maximal value.
Then
\begin{equation}
  -a_0 f(x_0)
  = \sum_{i=1}^{n-1} a_i f(x_i) - \nabla_n^2 f(x_0)
  \leq \sum_{i=1}^{n-1} a_i f(x_i)
  \leq \sum_{i=1}^{n-1} a_i f(x_0)
  = -a_0 f(x_0).
\end{equation}
As the RHS is equal to the LHS, the equality must hold throughout.
Since $f(x_i) \leq f(x_0)$ all points of the stencil must be equal, $f(x_1) = f(x_2) = \dots = f(x_n) = f(x_0)$.
Applying this same argument to each of the neighbours, and then to their neighbours and so on, we ultimately reach the conclusion that the same value is also attained on $\partial \Omega$, and we thus have a contradiction.
\end{proof}

\begin{lemma}\label{pde:lemma:bound}
  If $\exists ~ \phi_n > 0$ such that the proper stencil $\nabla_n^2 \phi_n = 1$ on $\Omega$, and $v$ is a descrete function that equals zero on $\partial \Omega$, then
  \begin{equation}
    \inorm{v} \leq \max_{\partial\Omega}(\lvert \phi_n \rvert)
    \inorm{\nabla_n^2 v}.
  \end{equation}
\end{lemma}
\begin{proof}

Let $\inorm{\nabla_n^2 v} = M$.
Then
$$
\nabla_n^2 (v + \phi_n M) = \nabla_n^2 v +  M \geq 0.
$$
$\nabla_n^2$ is a proper stencil, so by lemma \ref{pde:lemma:max} $\nabla_n^2 (v + \phi_n M)$ attains its maximum value on $\partial \Omega$.
Thus
$$
\inorm{v}
\leq \inorm{v + \phi_n M}
\leq \max_{\partial \Omega}(\lvert v + \phi_n M\rvert)
= \max_{\partial\Omega} (\lvert \phi_n \rvert) M
= \max_{\partial\Omega} (\lvert \phi_n \rvert) \inorm{\nabla_n^2 v}.
$$
\end{proof}

\begin{remark}
  For the five and nine point stencil, such functions exist.
  \begin{equation}
    \phi_5(x,y) = \frac14 \left(\left(x-\frac12\right)^2 + \left(y-\frac12\right)^2\right)
    \label{pde:equation:phi5}
  \end{equation}
  has the property $\nabla_5^2 \phi_5(x,y) = 1$, takes the values $0 \leq \phi_5(x,y) \leq 1/8$ on $\Omega$ and attains the maximum on $\partial \Omega$.
    
  \newcommand{\phinine}{\frac{1}{5} \left(\left(x-\frac{1}{2}\right)^4+\left(y-\frac{1}{2}\right)^4\right) - \frac{6}{5} \left(x-\frac{1}{2}\right)^2\left(y-\frac{1}{2}\right)^2 + \frac{1}{4} \left(\left(x-\frac{1}{2}\right)^2+\left(y-\frac{1}{2}\right)^2\right)}
  Similarly, the function
  \begin{equation}
  \phi_9(x,y) = \phinine
  \label{pde:equation:phi9}
  \end{equation}
  is such that $\nabla_9^2 \phi_9(x,y) = 1$, takes the values $0 \leq \phi_9(x,y) \leq 3/40$ on $\Omega$ and attains the maximum on $\partial \Omega$.
  Both are shown in \cref{pde:figure:phi}.
\end{remark}

\newcommand{\phidisplay}[3]{
\begin{tikzpicture}
\begin{axis}[
	width=8.2cm, height=7.5cm,
	zmin=0.0,zmax=0.13,
	xlabel=$x$, xtick distance=0.25,
	ylabel=$y$, ytick distance=0.25,
	ztick={0,#2, 1/8}, zticklabels={0,#2, 1/8},
	scaled ticks=false,
	grid,
	title={$#3$},
]
	\addplot3 [
		surf,
		domain=0:1,
	] {#1};
\end{axis}
\end{tikzpicture}
}

\begin{figure}[t]
\centering
\phidisplay{1/4*((x-1/2)^2+(y-1/2)^2)}{-1}{\phi_5(x,y)}
\phidisplay{1/5*((x-1/2)^4+(y-1/2)^4)-6/5*(x-1/2)^2*(y-1/2)^2+1/4*((x-1/2)^2+(y-1/2)^2)}{3/40}{\phi_9(x,y)}
\caption{\label{pde:figure:phi}The functions $\phi_5(x,y)$ and $\phi_9(x,y)$ range from $\phi_5(\frac12,\frac12)=\phi_9(\frac12,\frac12)=0$ at the center to $\phi_5(0,0) = 1/8$ and $\phi_9(0,0)=3/40$ at the boundaries of $[0,1]\times[0,1]$.}
\end{figure}

%% 3
\begin{theorem}[Five point stencil stability]
If $\nabla^2 u = f$ and $\nabla_5^2 v = f$, then
$$
\inorm{u - v} \leq Ch^4 \lvert D^4 v\rvert_\infty.
$$
\end{theorem}
\begin{proof}
By lemma \ref{pde:lemma:bound}
$$
\inorm{u - v} \leq \frac18 \inorm{\nabla_5^2 (u  - v)}.
$$
By Taylor expanding $u(x + h, y)$ and $u(x - h, y)$ we get
$$
\frac{1}{h^2} \delta_x^2 u = \partial_x^2 u + \frac{h^2}{12} \partial_x^4 + \mathcal{O}(h^4),
$$
and similarly for $y$.
Therefore we get, to order $h^2$
\begin{align*}
  \nabla_5^2 u
  &=\frac{1}{h^2} (\delta_x^2 + \delta_y^2) u\\
  &= (\partial_x^2 + \partial_y^2 + \frac{h^2}{12} \partial_x^4 + \frac{h^2}{12} \partial_y^4) u\\
  &\leq \nabla^2 u + \frac{h^2}{12} \max(\partial_x^4 u, \partial_y^4 u)\\
  &\leq \nabla^2 u + C h^2 \lvert D^4 u \rvert_\infty.
\end{align*}
Here $\lvert D^n u\rvert_\infty$ is to be understood as the maximal value of all $n$-index derivatives of $u$.
For example for $D^2$ this would be $u_{xx}, u_{xy}, u_{yx}, \text{and } u_{yy}$.
\end{proof}


%% 3
\begin{theorem}[Nine point stencil stability]
TODO verify it should be $F f$ not $f$.
If $\nabla^2 u = f$ and $\nabla_9^2 v = F f = f + \frac{h^2}{12} \nabla_5^2 f$, then
$$
\inorm{u - v} \leq Ch^2 \lvert D^6 v\rvert_\infty.
$$
\end{theorem}
\begin{proof}
By lemma \ref{pde:lemma:bound}
$$
\inorm{u - v} \leq \frac{3}{40} \inorm{\nabla_9^2 (u  - v)}.
$$
Taylor expanding $\nabla_9^2 u$, similarly to what was done in the five point stencil case, we get
\begin{align*}
  \nabla_9^2 u
  &= (\nabla^2
  + \frac{h^2}{12} \partial_x^4
  + \frac{h^2}{12} \partial_y^4
  + \frac{h^2}{6} \partial_x^2 \partial_y^2
  ) u + h^4 C \mathcal{O}(D^6 u)\\
  &= \nabla^2 u + \frac{h^2}{12} \nabla^2 f + C h^4 \mathcal{O}(D^6 u).
\end{align*}
Here $\mathcal{O}(D^n u)$ is to be understood as terms involving $n$-index derivatives of $u$.
Notice that
$$
\nabla^2 f = \nabla_5^2 f + C h^2 \mathcal{O}(D^4 f),
$$
and that $\mathcal{O}(D^4 f)
= \mathcal{O}(D^4 \nabla^2 u)
= \mathcal{O}(D^6 u)$.

Thus
\begin{align}
  \inorm{\nabla_9^2 (u  - v)}
  &= \inorm{Ch^4 \mathcal{O}(D^6u)}\\
  &\leq Ch^4 \inorm{D^6u}
\end{align}
\end{proof}



\subsection{The Fast Poisson Solver}
We are to solve the equation
$$
A U = F.
$$
Assuming that the eigenvectors of $A$ are a complete set, we may write
$$
F = a_1 y_1 + a_2 y_2 + ...,
$$
where $y_1, y_2, \dots$ are the eigenvectors of $A$.
(TODO: $U$ is also expanded in the eigenvectors?)
The solution $U$ is the of course
$$
U =
\frac{a_1}{\lambda_1} y_1
+ \frac{a_2}{\lambda_2} y_2
+ \dots,
$$
where $\lambda_i$ is the eigenvalue corresponding to $y_i$.
In general, however, this is not a viable way to solve the problem, as it requires knowing all the eigenvalues and eigenvectors, as well as finding the coefficients $a_i$.
However, for the set of eigenvectors in this problem, which are sines, we have a very efficient algorithm for computing the coefficients, the discrete fast sine transform.
We also have simple analytical expressions for the eigenvalues.

Written as matrix expressions
\begin{align}
  A U &= F\\
  S\Lambda S U &= F\\
  &\Rightarrow\\
  U &= S\Lambda^{-1} S F,
\end{align}
where we used the fact that $S^{-1} = S$.
Moreover, formulated more directly with regards to implementing the solver, we have that the solution is
$$
U = IFST(FST(F) / \Lambda),
$$
where $IFST, FST$ are the inverse and normal fast sine tranform.

\subsection{Demonstration of order}
\begin{figure}[h]
  \centering
  \begin{tikzpicture}
  \begin{loglogaxis}[
      title={Demonstration of order of the five and nine point stencil.},
      xlabel={$N = N_x = N_y$},
      ylabel={Absolute error, $\lVert u - v \rVert_\infty$},
      legend pos=south west,
    ]
    \addplot[mark=none, dashed] table[x={Ns}, y={five_roof}] {PDE/order.dat};
    \addlegendentry{$C h^2$};
    \addplot[mark=none] table[x={Ns}, y={nine_roof}] {PDE/order.dat};
    \addlegendentry{$C h^4$};

    \addplot table[x={Ns}, y={five}] {PDE/order.dat};
    \addlegendentry{Five point};

    \addplot table[x={Ns}, y={nine}] {PDE/order.dat};
    \addlegendentry{Nine point};
  \end{loglogaxis}
\end{tikzpicture}

  \caption{Error as a function discretization points $N$
    TODO: figure is to be updated with prettier figure
    TODO: Most appropriate with function of h or N?}
\end{figure}


\subsection{Solving the Biharmonic equation}
Consider the equation
\begin{align*}
  \nabla^4 u &= f(x, y) &(x, y) &\in \phantom{\partial}\Omega\\
  u = \nabla^2u &= 0  &(x, y) &\in \partial\Omega
\end{align*}
where $f(x, y) =
\left(
\sin \pi x
\sin \pi y
\right)^4
e^{-(x-0.5)^2 - (y-0.5)^2}
$.

Then
\begin{equation*}
	\hat{f}_{mn} =
	\left[ 2 \integral{\sin^4(\pi x) \sin(m \pi x) e^{-(x-1/2)^2}}{x}{0}{1}\right]
	\left[ 2 \integral{\sin^4(\pi y) \sin(n \pi y) e^{-(y-1/2)^2}}{y}{0}{1}\right].
\end{equation*}
The integrals do have analytical solutions, but they are not easy to compute.
We have computed them with the SAGE CAS suite.
Below is an easy-to-copy expression for the first bracket $ = I(m)$ in the equation above.
Then $\hat{f}_{mn} = I(m) \, I(n)$, and the analytical solution can be computed (but it has infinitely many terms)
\begin{lstlisting}
I(m) = 2 * integral(sin(pi*x)^4*sin(m*pi*x)*exp(-(x-1/2)^2), x, 0, 1)
>>> I(m) = 48*(pi^9*m^5*e^2 - 20*(pi^9*e^2 + 2*pi^7*e^2)*m^3 - (pi^9*m^5 - 20*(pi^9 + 2*pi^7)*m^3 + 16*(4*pi^9 + 15*pi^7 + 5*pi^5)*m)*(-1)^m + 16*(4*pi^9*e^2 + 15*pi^7*e^2 + 5*pi^5*e^2)*m)/(pi^10*m^10*e^3 - 20*(2*pi^10*e^3 - pi^8*e^3)*m^8 + 16384*pi^8*e^3 + 16*(33*pi^10*e^3 - 20*pi^8*e^3 + 10*pi^6*e^3)*m^6 + 40960*pi^6*e^3 - 320*(8*pi^10*e^3 - 7*pi^8*e^3 - 2*pi^4*e^3)*m^4 + 33792*pi^4*e^3 + 256*(16*pi^10*e^3 + 35*pi^6*e^3 + 20*pi^4*e^3 + 5*pi^2*e^3)*m^2 + 10240*pi^2*e^3 + 1024*e^3)
\end{lstlisting}

\begin{figure}
\begin{tikzpicture}
\begin{axis}[
	xmin=0, xmax=27,
	declare function={
		myfunc(\m)=48*(pi^9*(\m)^5*e^2 - 20*(pi^9*e^2 + 2*pi^7*e^2)*(\m)^3 - (pi^9*(\m)^5 - 20*(pi^9 + 2*pi^7)*(\m)^3 + 16*(4*pi^9 + 15*pi^7 + 5*pi^5)*(\m))*(-1)^(\m) + 16*(4*pi^9*e^2 + 15*pi^7*e^2 + 5*pi^5*e^2)*(\m))/(pi^10*(\m)^10*e^3 - 20*(2*pi^10*e^3 - pi^8*e^3)*(\m)^8 + 16384*pi^8*e^3 + 16*(33*pi^10*e^3 - 20*pi^8*e^3 + 10*pi^6*e^3)*(\m)^6 + 40960*pi^6*e^3 - 320*(8*pi^10*e^3 - 7*pi^8*e^3 - 2*pi^4*e^3)*(\m)^4 + 33792*pi^4*e^3 + 256*(16*pi^10*e^3 + 35*pi^6*e^3 + 20*pi^4*e^3 + 5*pi^2*e^3)*(\m)^2 + 10240*pi^2*e^3 + 1024*e^3);
	},
	xtick={1,6,...,26},
	minor x tick num=4,
	xlabel=$m$, title={$I(m)$},
]
\addplot [domain=1:26, samples=26, mark=*] {myfunc(x)};
\addplot [domain=0:27, samples=2, dashed] {0};
\end{axis}
\end{tikzpicture}
\end{figure}
