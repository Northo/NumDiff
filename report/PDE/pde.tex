\section{Biharmonic equation}
\label{sec:PDE}

\newtheorem{theorem}{Theorem}
\newtheorem{lemma}{Lemma}

Consider the inhomogeneous Biharmonic equation with clamped boundary conditions on the unit square $\Omega = [0, 1]^2$:
\begin{subequations}\label{eq:PDE}
  \begin{equation}
    \nabla^4 u = f \quad, (x, y) \in \Omega,
  \end{equation}
  \begin{equation}
    u = 0, \nabla^2u = 0 \quad, (x, y) \in \partial\Omega.
  \end{equation}
\end{subequations}

\subsection{Analytical solution}

Due to the single-valued Dirichlet boundary conditions, the solution $u(x,y)$ is periodic on $[0,1]\times[0,1]$ and can therefore be represented by a Fourier series $\sum_{m=-\infty}^{+\infty} \sum_{n=-\infty}^{+\infty} \hat{u}_{mn} \exp(i 2 \pi x) \exp(i 2 \pi y)$.
In particular, $u = 0$ at the boundary, so constants and cosines must drop out of the Fourier series.
It can therefore be represented by a double sine-only Fourier series (after rescaling $\hat{u}_{mn}$)
\newcommand{\fourierseries}[3]{
	\sum_{#2=1}^{\infty} \sum_{#3=1}^{\infty} #1 \sin(2 #2 \pi x) \sin(2 #2 \pi y)
}
\newcommand{\fourierexpand}[1]{
	\fourierseries{\hat{#1}_{mn}}{m}{n}
}
\begin{equation}
u(x,y) = \fourierexpand{u}.
\label{pde:equation:fourierexpansion}
\end{equation}

Plug the Fourier series into \cref{eq:PDE} and act with the biharmonic operator to get
\begin{equation*}
\nabla^4 u(x,y) = \fourierseries{((2m\pi)^2+(2n\pi)^2)^2 \hat{u}_{mn}}{m}{n} = f.
\end{equation*}
For simplicity, we \textbf{restrict ourselves to sources $f(x,y)$ that also vanish at the boundary}.
Then the source can also be written
\begin{equation}
	f(x,y) = \fourierexpand{f}.
\end{equation}

The Fourier series for $\nabla^4 u(x,y)$ and $f(x,y)$ can be equal only if their coefficients are equal.
This can be seen formally by multiplying both by $\sin(2m'\pi x) \sin(2n'\pi y)$, integrating over $x$ and $y$ and using the orthogonality of the sine functions,
\begin{equation*}
	\integral{\sin(2m\pi x) \sin(2m' \pi x)}{x}{0}{1} = \frac{1}{2} \delta_{mm'}.
\end{equation*}
Therefore, the coefficients of the solution are
\begin{equation}
	\hat{u}_{mn} = \frac{\hat{f}_{mn}}{((2m\pi)^2+(2n\pi)^2)^2},
\end{equation}
and the solution $u(x,y)$ is available by summing its Fourier series.

If we know $\hat{f}_{mn}$, it is straightforward to compute $\hat{u}_{mn}$ and thus the solution $u(x,y)$ itself from its Fourier series.
If we only know $f(x,y)$, we can find the coefficients by using the orthogonality of the sine functions again.
Multiply the Fourier series by $\sin(2 m' \pi x) \sin(2 n' \pi y)$ and integrate over $x$ and $y$ to get
\begin{equation*}
\begin{split}
  & \,\, \integraltwo{f(x,y) \sin(2 m' \pi x) \sin(2 n' \pi y)}{x}{y}{0}{1}{0}{1} \\
= & \,\, \sum_{m=1}^{\infty} \sum_{n=1}^{\infty} \hat{f}_{mn} \underbrace{\integral{\sin(2m\pi x)\sin(2m'\pi x)}{x}{0}{1}}_{\delta_{mm'}/2} \underbrace{\integral{\sin(2n\pi y)\sin(2n'\pi y)}{y}{0}{1}}_{\delta_{nn'}/2} \\
= & \,\, \hat{f}_{m'n'} / 4.
\end{split}
\end{equation*}
Read from bottom to top,
\begin{equation}
\hat{f}_{mn} = 4 \integraltwo{f(x,y) \sin(2 m \pi x) \sin(2 n \pi y)}{x}{y}{0}{1}{0}{1}.
\end{equation}

Note that a general source $f(x,y)$ may only be represented exactly by an infinite Fourier series.
To make the Fourier series solution viable, we must cut it off to include only a finite number of terms.
In this case, we should analyze $f(x,y)$ to make sure that we exclude only Fourier modes that contribute insignificantly to the solution.
However, we can also construct problems with a finite number of terms in the \emph{exact} solution by simply defining the source $f(x,y)$ in terms of a finite number of nonzero Fourier coefficients.

%% b)

\subsection{Numerical solution method}

We may transform \eqref{eq:PDE} into a system of Poisson equation by introducing $g = \nabla^2 u$,
\begin{align}\label{eq:PDE-poisson}
  \nabla^2g &= f,\\
  \nabla^2u &= g,\\
  g = u &= 0,\quad \text{ on } \partial \Omega.
\end{align}


%% c)
In solving the poisson equation $\nabla^2 u = f$ will consider here two discretization schemes, the five point stencil and nine point stencil, which we will denote $\nabla_5^2$ and $\nabla_9^2$.

\newcommand{\crossStencil}[5]{%
  \begin{tikzpicture}[scale=0.5,baseline=1mm, every node/.style={scale=0.7}, nodes={draw, circle}]
    \draw node[below]{$#1$} (0,0) -- (0,2) node[above]{$#2$};
    \draw (-1, 1) node[left]{$#3$} -- (1, 1) node[right]{$#4$};
    \node[above right, draw=none] at (0,1) {$#5$};
  \end{tikzpicture}
}

\newcommand{\xStencil}[4]{%
  \begin{tikzpicture}[scale=0.5,baseline=1mm, every node/.style={scale=0.7}, nodes={draw, circle}]
    \draw node[below left]{$#1$} (0,0) -- (1.41, 1.41) node[above right]{$#2$};
    \draw (0, 1.41) node[above left]{$#3$} -- (1.41, 0) node[below right]{$#4$};
  \end{tikzpicture}
}

Written in stencil diagrams, the five point stencil is given as 
$$
u \left(\crossStencil{1}{1}{1}{1}{-4}\right) = h^2 f
$$
while the nine point stencil is given by
$$
u
\left(
\xStencil{\frac16}{\frac16}{\frac16}{\frac16}
+
\crossStencil{\frac23}{\frac23}{\frac23}{\frac23}{-\frac{10}{3}}
\right)
=
h^2
f
\left(
\crossStencil{\frac1{12}}{\frac1{12}}{\frac1{12}}{\frac1{12}}{\frac23}
\right).
$$


We will now describe the structure of the matricies involved in the problem, starting with the simpler five point stencil.
We will use the same flattening for the 2D discrete function $u$ as described in section \ref{sec:exc3:numerical}.
We recognize that in $U$ neighbouring elements correspond to grid points that are left and right of each other, while the grid point up and down is the element N places before and after.
Thus, writing out the five point stencil $K2D$, we get a toeplitz and symmetric matrix, where the main diagonal has -4 and the $\pm 1$ and $\pm N$ off diagonals have 1.
We can compactley write this as
$$
K2D =
\begin{bmatrix}
  K & 0 \\
  0 & K & 0\\
  0 & 0 & K \\
  &&&\ddots
\end{bmatrix}
+
\begin{bmatrix}
  -2I & I & 0 &  \\
  I & -2I & I & 0 \\
  0 & I & -2I & I & 0 \dots\\
  \vdots&&&\ddots
\end{bmatrix}.
$$
where $I$ is the identity matrix and $K$ is the one dimensional central finite difference matrix of order 2,
$$
K =
\begin{bmatrix}
  -2 & 1 &   \\
  1 & -2 & 1 &  \\
  & 1 & -2 & 1 & \\
  &&&\ddots\\
  && 1 & -2 & 1\\
  &&& 1 & -2
\end{bmatrix}.
$$
Using the Kronecker product, this may be written as
$$
K2D = \verb|kron|(I, K) + \verb|kron|(K, I)
$$
which is useful when working with sparse matricies.

We will split the nine point stencil matrix into two, according to the stencil diagram above.
One matrix represents a five point stencil with weights altered to $\frac{-10}{3}, \frac23$ instead of $-4, 1$, we denote this matrix by $K2D^{(9)}$.
The other is the ``X''-stencil taking care of the NE, NW, SE, and SW points, each with a weight of $\frac16$, as shown in the diagram.
We will denote this matrix by $\Sigma 2D$.

Let
\begin{equation}
  \Sigma =
  \frac{1}{\sqrt{6}}
  \begin{bmatrix}
    0 & 1  \\
    1 & 0 & 1 \\
      & 1 & 0 & 1 \\
      &   & 1 & 0 & \ddots\\
      &   &   & \ddots  & \ddots
  \end{bmatrix}
\end{equation}
be the TST matrix with ones on its off-diagonals and zero on the diagonal.
Then we have $\Sigma 2D = \Sigma \otimes \Sigma$.
The nine point stencil is then
\begin{equation}
  somename9stencil = K2D^{(9)} + \Sigma 2D
\end{equation}

The eigenvalues of the Kronecker product $A \otimes B$ is the product of the eigenvalues of $A$ and $B$.
The eigenvalue of the Kronecker sum $A\oplus B = A \otimes I + I \otimes B$ is the sum of eigenvalues of $A$ and $B$.
In general the eigenvalues of a TST matrix are
\begin{equation*}
	\lambda_k = a + 2 b \cos\left(\frac{k \pi}{N + 1}\right)
\end{equation*}
where $a$ is the main diagonal and $b$ is the off diagonal.
%% Thus the eigenvalues of $K$ are
%% $$
%% \lambda_k = -2
%% \left(1 - \cos\left(\frac{k \pi}{N + 1}\right)\right)
%% $$
%% and the eigevalues of $\Sigma$ are
%% $$
%% \gamma_k = -\frac{2}{\sqrt{6}}\cos\left(\frac{k \pi}{N + 1}\right).
%% $$
Thus, the eigenvalues of $K2D$ are
$$
\left(-2 + 2\cos\left(\frac{k \pi}{N + 1}\right)\right)
+
\left(-2 + 2\cos\left(\frac{l \pi}{N + 1}\right)\right).
$$
The eigenvalues of $K2D^{(9)}$ are
$$
\frac13\left(-10 + 4\cos\left(\frac{k \pi}{N + 1}\right)\right)
+
\frac13\left(-10 + 4\cos\left(\frac{l \pi}{N + 1}\right)\right)
$$
and the eigenvalues of $\Sigma \otimes \Sigma$ are
$$
\frac46
\cos\left(\frac{k \pi}{N+1}\right)
\cos\left(\frac{l \pi}{N+1}\right).
$$
The eigenvalues of the nine point stencil are thus
\begin{equation}
  -\frac{10}{3}
  + \frac43
  \left(
  \cos(\frac{k \pi}{N+1})
  + \cos(\frac{l \pi}{N+1})
  \right)
  +
  \frac46
  \cos(\frac{k \pi}{N+1})
  \cos(\frac{l \pi}{N+1}).
\end{equation}

%% Let us defne an infinity norm for convenience
\newcommand{\inorm}[1]{
\lVert #1 \rVert_\infty
}

\subsection{Stability and order of the five point stencil}
We will here prove that the five point stencil is of order 2, following the outline provided in the exercise.
\emph{TODO: stability?}

%% 1
\begin{lemma}\label{pde:lemma:max}
If $\nabla_5^2 f \geq 0$ on $\Omega$, then the maximum value of $f$ is attained on $\partial \Omega$, that is
$$
\max(f)_\Omega \leq \max(f)_{\partial \Omega}.
$$
\end{lemma}
\begin{proof}
Suppose the opposite is true,
$$
\max(f)_\Omega > \max(f)_{\partial \Omega}.
$$
Then, there is an internal grid point on which $f$ attains it maximal value.
We denote this point by $x_0$, and label its neighbouring points $x_i, i = 1,2,3,4$.
Then
$$
4f(x_0) = \sum_{i=1}^4 f(x_i) - \nabla_5^2 f(x_0) \leq \sum_{i=1}^4 f(x_i) \leq 4 f(x_0),
$$
where the last inequality is attained from the fact that $x_0$ is assumed to be the maximal value of $f$.
As the RHS is equal to the LHS, the equality must hold throughout, and as $f(x_i) \leq f(x_0)$ we must have that all the neighbouring values have the same value as $x_0$, $f(x_1) = f(x_2) = f(x_3) = f(x_4) = f(x_0)$.
Applying this same argument to each of the neighbours, and then to their neighbours and so on, we ultimately reach the conclusion that all internal points have the same value, and that the same value is also attained on $\partial \Omega$, and we thus have a contradiction.
\end{proof}

%% 2
\begin{lemma}\label{pde:lemma:bound}
If $v$ is a discrete function that equals zero on $\partial \Omega$, then
$$
\inorm{v} \leq \frac18 \inorm{\nabla_5^2 v}.
$$
\end{lemma}
\begin{proof}
Consider the function
\begin{equation}
\phi_5(x,y) = \frac14 \left(\left(x-\frac12\right)^2 + \left(y-\frac12\right)^2\right).
\label{pde:equation:phi5}
\end{equation}
It has the property $\nabla_5^2 \phi_5(x,y) = 1$, takes the values $0 \leq \phi_5(x,y) \leq 1/8$ on $\Omega$ and attains the maximum on $\partial \Omega$, as shown in \cref{pde:figure:phi}.
Let $\inorm{\nabla_5^2 v} = M$.
Then
$$
\nabla_5^2 (v + \phi_5 M) = \nabla_5^2 v +  M \geq 0.
$$
By lemma \ref{pde:lemma:max} we have then that $\nabla_5^2 (v + \phi_5 M)$ attains its maximum value on $\partial \Omega$.
Thus
$$
\inorm{v}
\leq \inorm{v + \phi_5 M}
\leq \max_{\partial \Omega}(v + \phi_5 M)
= \max_{\partial\Omega} (\phi_5 M)
= \frac18 \inorm{\nabla_5^2 v}.
$$
\end{proof}

\newcommand{\phidisplay}[3]{
\begin{tikzpicture}
\begin{axis}[
	width=8.2cm, height=7.5cm,
	zmin=0.0,
	xlabel=$x$, xtick distance=0.25,
	ylabel=$y$, ytick distance=0.25,
	ztick={0,#2}, zticklabels={0,#2},
	scaled ticks=false,
	grid,
	title={$#3$},
]
	\addplot3 [
		surf,
		domain=0:1,
	] {#1};
\end{axis}
\end{tikzpicture}
}

\begin{figure}[t]
\centering
\phidisplay{1/4*((x-1/2)^2+(y-1/2)^2)}{1/8}{\phi_5(x,y)}
\phidisplay{1/5*((x-1/2)^4+(y-1/2)^4)-6/5*(x-1/2)^2*(y-1/2)^2+1/4*((x-1/2)^2+(y-1/2)^2)}{3/40}{\phi_9(x,y)}
\caption{\label{pde:figure:phi}The functions $\phi_5(x,y)$ and $\phi_9(x,y)$ range from $\phi_5(\frac12,\frac12)=\phi_9(\frac12,\frac12)=0$ at the center to $\phi_5(0,0) = 1/8$ and $\phi_9(0,0)=3/40$ at the boundaries of $[0,1]\times[0,1]$.}
\end{figure}

%% 3
\begin{theorem}
If $\nabla^2 u = f$ and $\nabla_5^2 v = f$, then
$$
\inorm{u - v} \leq Ch^4 \lvert D^4 v\rvert_\infty.
$$
\end{theorem}
\begin{proof}
By lemma \ref{pde:lemma:bound}
$$
\inorm{u - v} \leq \frac18 \inorm{\nabla_5^2 (u  - v)}.
$$
By Taylor expanding $u(x + h, y)$ and $u(x - h, y)$ we get
$$
\frac{1}{h^2} \delta_x^2 u = \partial_x^2 u + \frac{h^2}{12} \partial_x^4 + \mathcal{O}(h^4),
$$
and similarly for $y$.
Therefore we get, to order $h^2$
\begin{align*}
  \nabla_5^2 u
  &=\frac{1}{h^2} (\delta_x^2 + \delta_y^2) u\\
  &= (\partial_x^2 + \partial_y^2 + \frac{h^2}{12} \partial_x^4 + \frac{h^2}{12} \partial_y^4) u\\
  &\leq \nabla^2 u + \frac{h^2}{12} \max(\partial_x^4 u, \partial_y^4 u)\\
  &\leq \nabla^2 u + C h^2 \lvert D^4 u \rvert_\infty.
\end{align*}
Here $\lvert D^n u\rvert_\infty$ is to be understood as the maximal value of all $n$-index derivatives of $u$.
For example for $D^2$ this would be $u_{xx}, u_{xy}, u_{yx}, \text{and } u_{yy}$.
\end{proof}


\subsection{Stability and order of the nine point stencil}
Our proof for the nine point stencil will follow the same structure as for the five point stencil.

%% 1
\begin{lemma}\label{pde:lemma:max9}
If $\nabla_9^2 f \geq 0$ on $\Omega$, then the maximum value of $f$ is attained on $\partial \Omega$, that is
$$
\max(f)_\Omega \leq \max(f)_{\partial \Omega}.
$$
\end{lemma}
\begin{proof}
Suppose the opposite is true,
$$
\max(f)_\Omega > \max(f)_{\partial \Omega}.
$$
Then, there is an internal grid point on which $f$ attains it maximal value.
We denote this point by $x_0$, and label its neighbouring points $x_i, i = 1,2, \dots, 7, 8$, where the first four are the cardinal points and the latter four the NE, NW, SE, SW points.
Then
\begin{align*}
\frac{10}{3} f(x_0)
&= \frac{1}{6} \sum_{i=1}^4 f(x_i)
+ \frac{2}{3} \sum_{i=5}^8 f(x_i)
- \nabla_9^2 f(x_0)\\
&\leq \frac{1}{6} \sum_{i=1}^4 f(x_i)
+ \frac{2}{3} \sum_{i=5}^8 f(x_i)\\
&\leq
\left( 4\frac16 + 4\frac23 \right) u(x_0)\\
&= \frac{10}{3} u(x_0).
\end{align*}
By the same reasoning as in lemma \ref{pde:lemma:max} this leads to a contradiction.
\end{proof}

%% 2
\begin{lemma}\label{pde:lemma:bound9}
  If $v$ is a discrete function that equals zero on $\partial \Omega$,
  $$
  \inorm{v} \leq \frac{3}{40} \inorm{\nabla_9^2 v}.
  $$
\end{lemma}
\newcommand{\phinine}{\frac{1}{5} \left(\left(x-\frac{1}{2}\right)^4+\left(y-\frac{1}{2}\right)^4\right) - \frac{6}{5} \left(x-\frac{1}{2}\right)^2\left(y-\frac{1}{2}\right)^2 + \frac{1}{4} \left(\left(x-\frac{1}{2}\right)^2+\left(y-\frac{1}{2}\right)^2\right)}
\begin{proof}
  Consider the function
  \begin{equation}
  \phi_9(x,y) = \phinine
  \label{pde:equation:phi9}
  \end{equation}
  It has the property $\nabla_9^2 \phi_9(x,y) = 1$, takes the values $0 \leq \phi_9(x,y) \leq 3/40$ on $\Omega$ and attains the maximum on $\partial \Omega$, as shown in \cref{pde:figure:phi}.
  Repeating the proof of \cref{pde:lemma:bound} with $\phi_5(x,y) \rightarrow \phi_9(x,y)$ and $\nabla_5 \rightarrow \nabla_9$ shows that
  $$
  \inorm{v}
  \leq \frac{3}{40} \inorm{\nabla_9^2 v}.
  $$
\end{proof}

%% 3
\begin{theorem}
TODO verify it should be $F f$ not $f$.
If $\nabla^2 u = f$ and $\nabla_9^2 v = F f = f + \frac{h^2}{12} \nabla_5^2 f$, then
$$
\inorm{u - v} \leq Ch^2 \lvert D^6 v\rvert_\infty.
$$
\end{theorem}
\begin{proof}
By lemma \ref{pde:lemma:bound9}
$$
\inorm{u - v} \leq \frac{3}{40} \inorm{\nabla_9^2 (u  - v)}.
$$
Taylor expanding $\nabla_9^2 u$, similarly to what was done in the five point stencil case, we get
\begin{align*}
  \nabla_9^2 u
  &= (\nabla^2
  + \frac{h^2}{12} \partial_x^4
  + \frac{h^2}{12} \partial_y^4
  + \frac{h^2}{6} \partial_x^2 \partial_y^2
  ) u + h^4 C \mathcal{O}(D^6 u)\\
  &= \nabla^2 u + \frac{h^2}{12} \nabla^2 f + C h^4 \mathcal{O}(D^6 u).
\end{align*}
Here $\mathcal{O}(D^n u)$ is to be understood as terms involving $n$-index derivatives of $u$.
Notice that
$$
\nabla^2 f = \nabla_5^2 f + C h^2 \mathcal{O}(D^4 f),
$$
and that $\mathcal{O}(D^4 f)
= \mathcal{O}(D^4 \nabla^2 u)
= \mathcal{O}(D^6 u)$.

Thus
\begin{align}
  \inorm{\nabla_9^2 (u  - v)}
  &= \inorm{Ch^4 \mathcal{O}(D^6u)}\\
  &\leq Ch^4 \inorm{D^6u}
\end{align}
\end{proof}



\subsection{The Fast Poisson Solver}
We are to solve the equation
$$
A U = F.
$$
Assuming that the eigenvectors of $A$ are a complete set, we may write
$$
F = a_1 y_1 + a_2 y_2 + ...,
$$
where $y_1, y_2, \dots$ are the eigenvectors of $A$.
The solution $U$ is the of course
$$
U =
\frac{a_1}{\lambda_1} y_1
+ \frac{a_2}{\lambda_2} y_2
+ \dots,
$$
where $\lambda_i$ is the eigenvalue corresponding to $y_i$.
In general, however, this is not a viable way to solve the problem, as it requires knowing all the eigenvalues and eigenvectors, as well as finding the coefficients $a_i$.
However, for the set of eigenvectors in this problem, which are sines, we have a very efficient algorithm for computing the coefficients, the discrete fast sine transform.
We also have simple analytical expressions for the eigenvalues.

Written as matrix expressions
\begin{align}
  A U &= F\\
  S\Lambda S U &= F\\
  &\Rightarrow\\
  U &= S\Lambda^{-1} S F,
\end{align}
where we used the fact that $S^{-1} = S$.
Moreover, formulated more directly with regards to implementing the solver, we have that the solution is
$$
U = IFST(FST(F) / \Lambda),
$$
where $IFST, FST$ are the inverse and normal fast sine tranform.

\subsection{Demonstration of order}
\begin{figure}[h]
  \centering
  \begin{tikzpicture}
  \begin{loglogaxis}[
      title={Demonstration of order of the five and nine point stencil.},
      xlabel={$N = N_x = N_y$},
      ylabel={Absolute error, $\lVert u - v \rVert_\infty$},
      legend pos=south west,
    ]
    \addplot[mark=none, dashed] table[x={Ns}, y={five_roof}] {PDE/order.dat};
    \addlegendentry{$C h^2$};
    \addplot[mark=none] table[x={Ns}, y={nine_roof}] {PDE/order.dat};
    \addlegendentry{$C h^4$};

    \addplot table[x={Ns}, y={five}] {PDE/order.dat};
    \addlegendentry{Five point};

    \addplot table[x={Ns}, y={nine}] {PDE/order.dat};
    \addlegendentry{Nine point};
  \end{loglogaxis}
\end{tikzpicture}

  \caption{Error as a function discretization points $N$
    TODO: figure is to be updated with prettier figure
    TODO: Most appropriate with function of h or N?}
\end{figure}


\subsection{Solving the Biharmonic equation}
Consider the equation
\begin{align*}
  \nabla^4 u &= f(x, y) &(x, y) &\in \phantom{\partial}\Omega\\
  u = \nabla^2u &= 0  &(x, y) &\in \partial\Omega
\end{align*}
where $f(x, y) =
\left(
\sin \pi x
\sin \pi y
\right)^4
e^{-(x-0.5)^2 - (y-0.5)^2}
$.

