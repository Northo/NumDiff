\section{Biharmonic equation}
\label{sec:PDE}

Consider teh inhomogeneous Biharmonic equation with clamped boundary conditions on the unit square $\Omega = [0, 1]^2$:
\begin{subequations}\label{eq:PDE}
  \begin{equation}
    \nabla^4 u = f \quad, (x, y) \in \Omega,
  \end{equation}
  \begin{equation}
    u = 0, \nabla^2u = 0 \quad, (x, y) \in \partial\Omega.
  \end{equation}
\end{subequations}

We will begin by showing that the solution can be written as a double Fourier sine series
\begin{equation}\label{eq:sin-series}
  u(x, y) =
  \sum_{m=1}^\infty
  \sum_{n=1}^\infty
  F_{mn} \sin(m\pi x)\sin(n\pi y).
\end{equation}

Take \eqref{eq:sin-series} as an Ansatz.
Then it follow that
\begin{align}
  \nabla^4 u &=
  \sum_{m=1}^\infty
  \sum_{n=1}^\infty
  F_{mn}
  \nabla^4
  \sin(m\pi x)\sin(n\pi y)\\
  &=
  \sum_{m=1}^\infty
  \sum_{n=1}^\infty
  F_{mn}
  \left(
  (m\pi)^4 + (n\pi)^4 + 2n^2m^2\pi^4
  \right)
  \sin(m\pi x)\sin(n\pi y)
  &= f.
\end{align}

%% b)

We may transform \eqref{eq:PDE} into a system of Poisson equation by introducing $g = \nabla^2 u$,
\begin{align}\label{eq:PDE-poisson}
  \nabla^2g &= f,\\
  \nabla^2u &= g.
\end{align}


%% c)
We will consider here two approximations of $\nabla^2 u = f$, the five point stencil and nine point stencil, which we will denote $\nabla_5^2$ and $\nabla_9^2$.

\newcommand{\crossStencil}[5]{%
  \begin{tikzpicture}[scale=0.5,baseline=1mm, every node/.style={scale=0.7}]
    \draw node[below]{$#1$} (0,0) -- (0,2) node[above]{$#2$};
    \draw (-1, 1) node[left]{$#3$} -- (1, 1) node[right]{$#4$};
    \node[above right] at (0,1) {$#5$};
  \end{tikzpicture}
}

\newcommand{\xStencil}[4]{%
  \begin{tikzpicture}[scale=0.5,baseline=1mm, every node/.style={scale=0.7}]
    \draw node[below left]{$#1$} (0,0) -- (1.41, 1.41) node[above right]{$#2$};
    \draw (0, 1.41) node[above left]{$#3$} -- (1.41, 0) node[below right]{$#4$};
  \end{tikzpicture}
}

Written in stencil diagrams, the five point stencil is given as
$$
u
\left(\crossStencil{1}{1}{1}{1}{-4}\right) = h^2 f
$$
while the nine point stencil is given by
$$
u
\left(
\xStencil{\frac16}{\frac16}{\frac16}{\frac16}
+
\crossStencil{\frac23}{\frac23}{\frac23}{\frac23}{-\frac{10}{3}}
\right)
=
h^2
f
\left(
\crossStencil{\frac1{12}}{\frac1{12}}{\frac1{12}}{\frac1{12}}{\frac23}
\right).
$$

Using one of these stencils will give a system of equations which we can write as a matrix equation,
$$
AU = BF,
$$
where $A$ and $B$ are matricies, $U$ is the solution that we seek, and $F$ is the inhomogeneity.
More precisely $U$ is a flattened array which represents the solution $u$;
the flattening is such that the rows of two-dimensional version of $U$ are conserved.
Thus, the next element of a column at position $n$, is at position $n+N$ in the flattened array, where $N$ is the number of elements in a row.
We will now consider the structure of the matrix $A$ in more detail.

We start with the simpler five point stencil.
blablabla

We can compactley write this as
$$
K2D =
\begin{bmatrix}
  K & 0 \\
  0 & K & 0\\
  0 & 0 & K \\
  &&&\ddots
\end{bmatrix}
+
\begin{bmatrix}
  -2I & I & 0 &  \\
  I & -2I & I & 0 \\
  0 & I & -2I & I & 0 \dots\\
  \vdots&&&\ddots
\end{bmatrix}.
$$
where $I$ is the identity matrix and $K$ is the one dimensional central finite difference matrix,
$$
K =
\begin{bmatrix}
  -2 & 1 & 0 &  \\
  1 & -2 & 1 & 0 \\
  0 & 1 & -2 & 1 & 0 \dots\\
  \vdots&&&\ddots
\end{bmatrix}.
$$
Using the Kronecker product, this may be written as
$$
K2D = \verb|kron|(I, K) + \verb|kron|(K, I)
$$

The nine point stencil has a slightly more complicated structure.
The block structure of the nine point stencil is as follows:
$$
\begin{bmatrix}
  4Q - 36 I & Q \\
  Q & 4Q - 36 I & 4Q \\
  0 & Q & 4Q - 36 I & Q \\
  & & & \ddots
\end{bmatrix}
=
\begin{bmatrix}
  4Q & Q \\
  Q & 4Q & 4Q \\
  0 & Q & 4Q & Q \\
  & & & \ddots
\end{bmatrix}
-
36
\begin{bmatrix}
  I & 0 \\
  0 & I & 0 \\
  0 & 0 & I & 0 \\
  & & & \ddots
\end{bmatrix}
$$

Notice the interesting recursive structure.

We can also write the nine point stencil as a sum of a five point stencil, where we simply change the weights, another matrix taking care of the NE, NW, etc. points.

Let
\begin{equation}
  \Sigma =
  \frac{1}{\sqrt{6}}
  \begin{bmatrix}
    0 & 1  \\
    1 & 0 & 1 \\
      & 1 & 0 & 1 \\
      &   & 1 & 0 & \ddots\\
      &   &   & \ddots  & \ddots
  \end{bmatrix}
\end{equation}
be the TST matrix with ones on its off-diagonals and zero on the diagonal.
The nine point stencil is then
\begin{equation}
  K2D^{(9)} + \Sigma \otimes \Sigma.
\end{equation}

We find the eigenvalues of $K2D^{(9)}$ directly by substituting in the appropriate weights in our previous expression from the five point stencil.
The $\Sigma$ matrix must by its structure obviously have the same eigenvectors as $K$, and by substituting in the formula, interpreting the zero diagonal as a zero weight, we find that the eigenvalues of $\Sigma$ are
$$
\frac{2}{\sqrt{6}} \cos(\frac{k \pi}{N+1}).
$$
The eigenvalues of the Kronecker product $A \otimes B$ is the product of the eigenvalues of $A$ and $B$.
Thus, the eigenvalues of $\Sigma \otimes \Sigma$ are
$$
\frac46
\cos(\frac{k \pi}{N+1})
\cos(\frac{l \pi}{N+1}).
$$
The eigenvalues of the nine point stencil are thus
\begin{equation}
  -\frac{10}{3}
  + \frac43
  \left(
  \cos(\frac{k \pi}{N+1})
  + \cos(\frac{l \pi}{N+1})
  \right)
  +
  \frac46
  \cos(\frac{k \pi}{N+1})
  \cos(\frac{l \pi}{N+1}).
\end{equation}

\subsection{The Fast Poisson Solver}
We are to solve the equation
$$
A U = F.
$$
Assuming that the eigenvectors of $A$ are a complete set, we may write
$$
F = a_1 y_1 + a_2 y_2 + ...,
$$
where $y_1, y_2, \dots$ are the eigenvectors of $A$.
The solution $U$ is the of course
$$
U =
\frac{a_1}{\lambda_1} y_1
+ \frac{a_2}{\lambda_2} y_2
+ \dots,
$$
where $\lambda_i$ is the eigenvalue corresponding to $y_i$.
In general, however, this is not a viable way to solve the problem, as it requires knowing all the eigenvalues and eigenvectors, as well as finding the coefficients $a_i$.
However, for the set of eigenvectors in this problem, which are sines, we have a very efficient algorithm for computing the coefficients, the discrete fast sine transform.
We also have simple analytical expressions for the eigenvalues.

Written as matrix expressions
\begin{align}
  A U &= F\\
  S\Lambda S U &= F\\
  &\Rightarrow\\
  U &= S\Lambda^{-1} S F,
\end{align}
where we used the fact that $S^{-1} = S$.
Moreover, formulated more directly with regards to implementing the solver, we have that the solutin is
$$
U = IFST(FST(F) / \Lambda),
$$
where $IFST, FST$ are the inverse and normal fast sine tranform.
