\section{Linearized Korteweg-De Vries equation in one dimension}

In this section, we will study the one-dimensional linearized Korteweg-De Vries equation
\begin{equation}
\pd{u}{t} + \left(1+\pi^2\right)\pd{u}{x} + \pd[3]{u}{x} = 0 \qquad (t \geq 0) \quad (-L/2 \leq x \leq +L/2),
\label{kdv_equation}
\end{equation}
where the solution $u = u(x,t)$ is subject to periodic boundary conditions
\begin{equation*}
u(x+L, t) = u(x, t).
\end{equation*}

\subsection{Analytical solution}

As the solution is periodic in space at every instant $t$, it can be expressed as a Fourier-series
\begin{equation}
u(x, t) = \sum_{n=-\infty}^{+\infty} c_n(t) \exp{(i k_n x)}
\label{fourier_series}
\end{equation}
with wavenumbers $k_n = 2 \pi n / L$ and time-dependent coefficients $c_n(t)$ that ensure spatial periodicity at all times.
This can be derived formally by separation of variables.
Inserting the Fourier series into the equation gives
\begin{equation*}
	\sum_n \left( \dot{c}_n(t) + i \left( \left( 1+\pi^2 \right) k_n - k_n^3 \right) c_n(t) \right) \exp{(i k_n x)} = 0.
\end{equation*}
Due to orthogonality of the Fourier basis functions $\exp(i k_n x)$, this can only vanish if all coefficients separately vanish, so
\begin{equation}
	c_n(t) = c_n(0) \exp{\left( -i \left (\left(1+\pi^2\right)k_n - k_n^3\right) t \right) }.
	\label{fourier_coefficients}
\end{equation}

\subsection{Numerical solution method}

GRID FIGURE?? $x_0$ to $x_{M-1}$ makes modulo arithmetic intuitive.

We will solve the Korteweg-De Vries equation with the central finite differences
\begin{equation*}
\begin{split}
	\dpd{u}{x}    = \frac{\delta}{2h}       + O(h^2) &= \frac{u_{m+1}^n-u_{m-1}^n}{2h} + O(h^2) \\
	\dpd[3]{u}{x} = \frac{\delta^3}{(2h)^3} + O(h^2) &= \frac{u_{m+3}^n-3u_{m+1}^n+3u_{m-1}^n-u_{m-3}^n}{8h^3} + O(h^2)
\end{split}
\end{equation*}
in space, and doing the integration of $u_t = F(u_x, u_{xxx})$ in time using both the Euler method and the Crank-Nicholson method.
They can be written simultaneously by inserting $\theta = 0$ and $\theta = 1/2$, respectively, in the Theta method
\begin{equation*}
% \frac{u_m^{n+1} - u_m^n}{k} = (1-\theta) F(u^n) + \theta F(u^{n+1})
% TODO: mean really mean F(u_x, u_xxx)
% \frac{u(x,t+k)-u(x,t)}{k} = (1-\theta) F\left(u(x, t)\right) + \theta F\left(u(x, t+k)\right) 
% + O \left( k \left(\frac{1}{2}-\theta\right) + k^2 \left( \frac{1}{6} - \frac{\theta}{2} \right) \right).
\frac{U_m^{n+1} - U_m^n}{\Delta t} = (1-\theta) F(U_m^n) + \theta F(U_m^{n+1}),
\end{equation*}
where all spatial discretization is contained in
\begin{equation*}
\begin{split}
	F(U_m^n) &= -\left(1+\pi^2\right) \frac{\delta}{2h} U_m^n        - \frac{\delta^3}{(2h)^3} U_m^n \\
			 &= -\left(1+\pi^2\right) \frac{u_{m+1}^n-u_{m-1}^n}{2h} - \frac{u_{m+3}^n-3u_{m+1}^n+3u_{m-1}^n-u_{m-3}^n}{8h^3}
\end{split}
\end{equation*}
This results in the system of equations
\begin{equation}\label{theta_method_discretized}
	\left(1-\theta k \left(-\left(1+\pi^2\right)\frac{\delta}{2h} - \frac{\delta^3}{(2h)^3}\right)\right) U_m^{n+1}
	= \left(1-\left(1-\theta\right) k \left(-\left(1+\pi^2\right)\frac{\delta}{2h} - \frac{\delta^3}{(2h)^3}\right)\right) U_m^n
\end{equation}
for the unknown values $U_0^{n+1}, \,\dots\,, U_{M-1}^{n+1}$ at the next time step in terms of the known values $U_0^{n}, \,\dots\,, U_{M-1}^{n}$ at the current time step.
In matrix form, the system can be written
\begin{equation}
	\left(I - \theta k A\right) U^{n+1} = \left(I - \left(1-\theta\right) k A\right) U^n,
	\label{matrixeq}
\end{equation}
where $U^{n} = \begin{bmatrix} U_0^n & \dots & U_{M-1}^n \end{bmatrix}^T$ and $A = $
\newcommand\ca{\color{red}}
\newcommand\cb{\color{magenta}}
\newcommand\cc{\color{blue}}
\newcommand\cd{\color{cyan}}
% https://tex.stackexchange.com/a/571702
\begin{equation*}
\renewcommand{\arraystretch}{1.25} % stretch matrix vertically to make it square
\renewcommand{\arraycolsep}{4.7pt} % juuuust make matrix fill page width
% TODO: fix signs
\frac{-1}{2h}
\begin{bmatrix}
0           & \ca +1   &             &             &             &             &             &             & \cb -1      \\
\cb -1      & 0        & \ca +1      &             &             &             &             &             &             \\
            & \cb -1   & 0           & \ca +1      &             &             &             &             &             \\
            &          & \cb -1      & 0           & \ca +1      &             &             &             &             \\
            &          &             & \cb \ddots  & \ddots      & \ca \ddots  &             &             &             \\
            &          &             &             & \cb -1      & 0           & \ca +1      &             &             \\
            &          &             &             &             & \cb -1      & 0           & \ca +1      &             \\
            &          &             &             &             &             & \cb -1      & 0           & \ca +1      \\
\ca +1      &          &             &             &             &             &             & \cb -1      & 0           \\
\end{bmatrix}
-\frac{1+\pi^2}{h^3}
\begin{bmatrix}
% 0      & +1/2h  &        &        & -1/2h  \\
% -1/2h  & 0      & +1/2h  &        &        \\
       % & \ddots & \ddots & \ddots &        \\
       % &        & -1/2h  & 0      & +1/2h  \\
% +1/2h  &        &        & -1/2h  & 0      \\
0           & \ca -3      & 0           & \cc +1      &             &             & \cd -1      & 0           & \cb +3      \\
\cb +3      & 0           & \ca -3      & 0           & \cc +1      &             &             & \cd -1      & 0           \\
0           & \cb +3      & 0           & \ca -3      & 0           & \cc +1      &             &             & \cd -1      \\
\cd -1      & 0           & \cb +3      & 0           & \ca -3      & 0           & \cc +1      &             &             \\
            & \cd \ddots  & \ddots      & \cb \ddots  & \ddots      & \ca \ddots  & \ddots      & \cc \ddots  &             \\
            &             & \cd -1      & 0           & \cb +3      & 0           & \ca -3      & 0           & \cc +1      \\
\cc +1      &             &             & \cd -1      & 0           & \cb +3      & 0           & \ca -3      & 0           \\
0           & \cc +1      &             &             & \cd -1      & 0           & \cb +3      & 0           & \ca -3      \\
\ca -3      & 0           & \cc +1      &             &             & \cd -1      & 0           & \cb +3      & 0           \\
\end{bmatrix}
.
\end{equation*}
where we have imposed periodic boundary conditions $U_m^n = U_{m+M}^n$ by simply wrapping the spatial derivative stencils around the matrix.

We then solve the system by preparing $U^0$ from the initial condition and solve \cref{matrixeq} repeatedly to step forward in time.
Note that with the constant time step $k$, all matrices in \cref{matrixeq} are constant in time, and the process of solving the system many times can be accelerated by for example LU-factorizing the matrix on the left side.

\begin{figure}
\begin{tikzpicture}
\begin{axis}[
areaplot1/.style={fill opacity=0.50, fill=green, mark=none},
areaplot2/.style={fill opacity=0.50, fill=red, mark=none},
width=17cm,
height=12cm,
view={-28}{+30},
xlabel={$x$},
ylabel={$t$},
zmin=0,
ymajorgrids,
xmajorgrids,
zmajorgrids,
xtick distance=0.50,
ytick distance=0.25,
ztick distance=1,
legend cell align={left},
title={Crank-Nicholson ($M=800, \, N=100$)},
zticklabel={\pgfmathparse{\tick-1}$\pgfmathprintnumber{\pgfmathresult}$} % hack to offset labels by 1 (to make it possible to use filling, which is 0-based)
]
\pgfplotsinvokeforeach{5,4,...,1}{
	% TODO: set zlims correctly
	% Filled version
	% \addplot3 [areaplot1] table [x index=0,y expr={#1/5},z expr=\thisrowno{#1}+1] {exercise4/timeevol.dat} \closedcycle;
	% \pgfmathparse{int(round(5+#1))};
	% \pgfmathtruncatemacro\mymacro{round(5+#1)};
	% \addplot3 [areaplot2] table [x index=0,y expr={#1/5},z expr=\thisrowno{\mymacro}+1] {exercise4/timeevol.dat} \closedcycle;
	% \node[draw] at (0, 2) {\mymacro}; % for debug

	% Non-filled version
	\addplot3 [fill=gray, opacity=0.75, mark=none, draw=gray!80!black, thick] table [x index=0,y expr={(#1-1)/(5-1)},z expr=\thisrowno{#1}+1] {exercise4/timeevol.dat} \closedcycle;
	\ifthenelse{\equal{#1}{5}}{\addlegendentry{$u(x,t) = \sin(\pi(x-t))$}}{}
	\pgfmathparse{int(round(5+#1))};
	\pgfmathtruncatemacro\mymacro{round(5+#1)};
	\addplot3 [mark=none, color=red, thick] table [x index=0,y expr={(#1-1)/(5-1)},z expr=\thisrowno{\mymacro}+1] {exercise4/timeevol.dat};
	\ifthenelse{\equal{#1}{5}}{\addlegendentry{$U(x_m,t_n)$}}{}
	% \node[draw] at (0, 2) {\mymacro}; % for debug
}
\end{axis}
\end{tikzpicture}
\caption{\label{ex4_solution_3d}}
\end{figure}

Next, we test our numerical solution on the problem defined by the initial condition $u(x, 0) = \sin(\pi x)$ on $x \in [-1, +1]$ with $L = 2$.
The Fourier series of the analytical solution then has nonzero coefficients $c_{\pm1}(0) = \pm 1/2i$ and wavenumbers $k_{\pm 1} = \pm \pi$, which give rise to the analytical solution $u(x, t) = \sin(\pi(x-t))$.
As shown in \cref{ex4_solution_3d}, the solution represents a sine wave traveling with velocity $1$ to the right.

In \cref{ex4_euler_vs_crank_snapshots}, we compare snapshots of the numerical solution at $t = 1$ from the Euler method and the Crank-Nicholson method.
Note that the Crank-Nicholson method approaches the exact solution with as little as $N=10$ time steps and a few hundred spatial grid points $M$, while the Euler method produces garbage with as many as $N=300000$ time steps and seems to become unstable for very few spatial grid points.
\Cref{ex4_euler_vs_crank_convergence} supports our suspicions, showing the stable second order (?) nature of the Crank-Nicholson method and the unstability of the Euler method.

\begin{figure}
\begin{tikzpicture}
\begin{axis}[ymin=-1.2, ymax=+1.2, no markers,title={Forward Euler ($N=300000$)}]
\pgfplotsinvokeforeach{5,10,15,20,25} {
	\addplot table {exercise4/snapshot-forward-euler-M#1-N300000.dat};
	\addlegendentry{$M=#1$};
}
\addplot [xlabel=$x$, color=black, domain=-1:+1, very thick] {sin(deg(pi*(x-1)))};
\end{axis}
\end{tikzpicture}
\begin{tikzpicture}
\begin{axis}[ymin=-1.2, ymax=+1.2, no markers,title={Crank-Nicholson ($N=10$)}]
\pgfplotsinvokeforeach{200,400,600} {
	\addplot table {exercise4/snapshot-crank-nicholson-M#1-N10.dat};
	\addlegendentry{$M=#1$};
}
\addplot [xlabel=$x$, color=black, domain=-1:+1, very thick] {sin(deg(pi*(x-1)))};
\end{axis}
\end{tikzpicture}
\caption{\label{ex4_euler_vs_crank_snapshots}}
\end{figure}

\begin{figure}
\centering
\begin{tikzpicture}
\begin{loglogaxis}[xlabel=$M$,ylabel={relative error},legend pos=north east,width=8cm,title=Crank-Nicholson]
\pgfplotsinvokeforeach{10,20,30,40,50} {
	\addplot table [x=M, y=E#1] {exercise4/convergence-crank-nicholson.dat};
	\addlegendentry{$N=#1$};
}
\end{loglogaxis}
\end{tikzpicture}
\begin{tikzpicture}
\begin{loglogaxis}[xlabel=$M$,legend pos=north west,width=8cm,title=Forward Euler]
\pgfplotsinvokeforeach{10000,20000,30000} {
	\addplot table [x=M, y=E#1] {exercise4/convergence-forward-euler.dat};
	\addlegendentry{$N=#1$};
}
\end{loglogaxis}
\end{tikzpicture}
\caption{\label{ex4_euler_vs_crank_convergence}Convergence plot at $t=1$, comparison between Crank-Nicholson and Forward Euler}
\end{figure}

\subsection{Stability analysis}

Motivated by the above examples of the Euler method and the Crank-Nicholson method, we perform a Von Neumann analysis of their stability.
Just like the exact solution, the numerical solution is subject to periodic boundary conditions in space and can therefore be expanded in a Fourier series
\begin{equation}
	U_m^n = U(x_m, t_n) = \sum_l C_l^n \exp \left(i k_l x_m\right).
\end{equation}
Consider now only a single Fourier mode $C_l^n \exp (i k_l x_m)$ in the series.
Inserting it into \cref{theta_method_discretized}, dividing by $\exp(i k_l x_m)$ and expanding exponentials using Euler's identity gives
% TODO: need a cleaner version of the discretized equation
\begin{equation*}
% \frac{C_l^{n+1}-C_l^n}{\Delta t} = \left(\left(1-\theta\right)C_l^n+\theta C_l^{n+1}\right) \left(-(1+\pi^2) \frac{e^{i k_l h}-e^{-i k_l h}}{2h} - \frac{e^{3i k_l h}-e^{-3i k_l h}-3(e^{i k_l h}-e^{-i k_l h})}{8h^3}\right)
% \frac{C_l^{n+1}-C_l^n}{\Delta t} = i \left(\left(1-\theta\right)C_l^n+\theta C_l^{n+1}\right) \left(-(1+\pi^2) \frac{\sin(qh)}{h} - \frac{\sin^3(qh)}{h^3}\right)
\frac{C_l^{n+1}-C_l^n}{\Delta t} = i \left(\left(1-\theta\right)C_l^n+\theta C_l^{n+1}\right), \quad \text{where} \,\, f(k_l) = \left(-\left(1+\pi^2\right) \frac{\sin(k_l h)}{h} - \frac{\sin^3(k_l h)}{h^3}\right).
\end{equation*}

Now look at the amplification factor $G_l = C_l^{n+1} / C_l^n$ of Fourier mode $l$ over one time step. 
With $\theta = 1/2$, the Crank-Nicholson method gives 
\begin{equation*}
	G_l = \frac{1 + i \Delta t f(x) / 2}{1 - i \Delta t f(x) / 2} \quad \implies \quad \abs{G_l} = 1.
\end{equation*}
The amplitude of all Fourier modes is thus preserved over time independently of $\Delta t$ and $\Delta x$, and we say the Crank-Nicholson method is \textbf{unconditionally stable}.

The Euler method has $\theta = 0$ and gives
\begin{equation*}
	G_l = 1 + i \Delta t f(x) \quad \implies \quad \abs{G_l} = \sqrt{1 + \Delta t^2 f(k_l)^2}.
\end{equation*}
Since $\abs{\sin(k_l h)} \leq 1$ for all $k_l$, we can bound $f(k_l)$ by
\begin{equation*}
	\abs{f(k_l)} \leq \frac{(1+\pi^2)}{\Delta x} + \frac{1}{\Delta x^3} = \frac{1}{\Delta x^3} \left( (1+\pi^2) \Delta x^2 + 1 \right) \leq \frac{1}{\Delta x^3} \left( (1+\pi^2) L^2 + 1 \right).
\end{equation*}
Then $\abs{G_l} = \sqrt{1 + O(\Delta t^2 / \Delta x^6)}$.
% TODO: cite von neumann stability criterion
The Von Neumann stability criterion $\abs{G_l} \leq 1 + O(\Delta t)$ \cite{owren} is attained only with $\Delta t \geq O(\Delta x^6)$, corresponding to \emph{extremely small} time steps.
Thus, while the Euler method in theory is \textbf{conditionally stable}, it is unstable for all practical time steps.
Other methods are far superior, as they allow for both a much greater resolution of the spatial grid in addition to larger time steps.

% TODO: one figure with Euler, small M large N
% TODO: one figure with CN, 

\begin{figure}
\centering
% https://tex.stackexchange.com/questions/232690/color-coded-plots-with-colorbar-in-pgfplots
\begin{tikzpicture}
\begin{axis}[title={$N=10$},ylabel=$u$, xlabel=$x$, width=13cm, height=8cm, colorbar, colorbar style={ylabel=$M$}, xmin=-1, xmax=+1]
\addplot [color=black, domain=-1:+1, very thick] {sin(deg(pi*(x-1)))};
\addlegendentry{$\sin(\pi(x-1))$};
\pgfplotsinvokeforeach{20,40,60,80,100,200,300,400,500,600,700,800,900,1000} {
	\addplot [mesh, point meta=#1] table {exercise4/snapshot-crank-nicholson-M#1-N10.dat};
}
\end{axis}
\end{tikzpicture}
\begin{tikzpicture}
\begin{axis}[title={$N=100$},xlabel=$x$, width=13cm, height=8cm, colorbar, colorbar style={ylabel=$M$}, xmin=-1, xmax=+1]
\addplot [xlabel=$x$, color=black, domain=-1:+1, very thick] {sin(deg(pi*(x-1)))};
\addlegendentry{$\sin(\pi(x-1))$};
\pgfplotsinvokeforeach{20,40,60,80,100,200,300,400,500,600,700,800,900,1000} {
	\addplot [mesh, point meta=#1] table {exercise4/snapshot-crank-nicholson-M#1-N100.dat};
}
\end{axis}
\end{tikzpicture}
\end{figure}

\begin{figure}
\begin{tikzpicture}
\begin{axis}[
	height=10cm, width=1.00\textwidth, xlabel={$x$}, ylabel={$t$}, restrict z to domain=-5:+5, view={-10}{+10}, ymin=0, ymax=1,
]
	% \addplot3 [surf, mesh/cols=20] table {exercise4/cranknich.dat};
	\addplot3 [surf, mesh/cols=20] table {exercise4/crank-nicholson-M20-N400.dat};
\end{axis}
\end{tikzpicture}
\end{figure}

\subsection{Time evolution of norm}

The stability of the finite difference methods can be even better illustrated by investigating the time evolution of the $L_2$-norm of the solution.
To this end, we will first show that the $L_2$-norm of the analytical solution is preserved over time.
Then we will show the time evolution of the $L_2$-norm of numerical solutions.

The $L_2$-norm of the analytical solution is defined as
\begin{equation*}
	\Vert u(x,t) \Vert_2 = \left( \frac{1}{2} \int_{-L/2}^{+L/2} |u(x,t)|^2 \dif x \right)^{1/2}.
\end{equation*}
To understand why it is constant in time, insert the Fourier expansion \cref{fourier_series} for $u(x,t)$ to get
\begin{equation*}
\int_{-1}^{+1} \dif x \left| u(x,t) \right|^2 = \sum_{m,n} c_m(t) c_n^*(t) \underbrace{\int_{-L/2}^{+L/2} \exp{\left(i(k_m-k_n)x\right)}}_{L \delta_{m n}} \dif x = L \sum_n \left| c_n(t) \right|^2.
\end{equation*}
Since the Fourier coefficients \cref{fourier_coefficients} have constant magnitude $|c_n(t)| = |c_n(0)|$, the $L_2$-norm is the same at any time $t$.
Note in retrospect how the odd number of spatial derivatives in \cref{kdv_equation} ensures that the exponential in \cref{fourier_coefficients} has a purely imaginary argument and thus constant magnitude.

\begin{figure}
\centering
\begin{tikzpicture}
\begin{groupplot}[
	group style={group size=2 by 1, horizontal sep=0.3cm},
	height=7cm,
	width=0.52\textwidth,
	ymin=0.4, ymax=0.5, restrict y to domain=0:10,
]
	\pgfplotscreateplotcyclelist{mycycle}{
		{black,solid},
		{black,dashdotted},
		{black,dashed},
		{blue,solid},
		{blue,dashdotted},
		{blue,dashed},
		{green,solid},
		{green,dashdotted},
		{green,dashed},
		{red,solid},
		{red,dashdotted},
		{red,dashed},
	};

	\nextgroupplot[title={Forward Euler}, xlabel=$t$,ylabel={$\Vert U(t=1) \Vert_2$},cycle list name=mycycle,legend pos=north east];
	\addplot {0};
	\addlegendentry{$N=30000$};
	\addplot {0};
	\addlegendentry{$N=40000$};
	\addplot {0};
	\addlegendentry{$N=50000$};

	\pgfplotsinvokeforeach{15,20,25} {
		\addplot table [x expr=\coordindex/100, y={forward-euler-M#1-N30000}] {exercise4/norm-evolution.dat};
		\addplot table [x expr=\coordindex/100, y={forward-euler-M#1-N40000}] {exercise4/norm-evolution.dat};
		\addplot table [x expr=\coordindex/100, y={forward-euler-M#1-N50000}] {exercise4/norm-evolution.dat};
	}

	\nextgroupplot[title={Crank-Nicholson}, xlabel=$t$,yticklabels={,,}];
	\pgfplotsinvokeforeach{800} {
		\addplot [color=black] table [x expr=\coordindex/100, y={crank-nicholson-M#1-N100}] {exercise4/norm-evolution.dat};
		\addplot [color=black] table [x expr=\coordindex/100, y={crank-nicholson-M#1-N200}] {exercise4/norm-evolution.dat};
		\addplot [color=black] table [x expr=\coordindex/100, y={crank-nicholson-M#1-N300}] {exercise4/norm-evolution.dat};
	}
% \addplot [color=black] table [y={crank-nicholson-M800-N300}] {exercise4/norm-evolution.dat};
\end{groupplot}
\end{tikzpicture}
\end{figure}

Theta method:
\begin{equation}
\frac{u_m^{n+1} - u_m^n}{k} = (1-\theta) F(u^n) + \theta F(u^{n+1})
\end{equation}
where
\begin{equation}
F(u^n) = -(1+\pi^2) \frac{u_{m+1}^n-u_{m-1}^n}{2h} - \frac{u_{m+3}^n-3u_{m+1}^n+3u_{m-1}^n-u_{m-3}^n}{8h^3}
\end{equation}
$\theta = 0$ is forward Euler, $\theta = 1/2$ is Crank-Nicholson, $\theta = 1$ is backward Euler.
