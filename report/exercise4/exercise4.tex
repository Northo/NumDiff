\section{Linearized Korteweg-de Vries equation in one dimension}

In this section, we will study the one-dimensional linearized Korteweg-De Vries equation
\begin{equation}
\pd{u}{t} + \left(1+\pi^2\right)\pd{u}{x} + \pd[3]{u}{x} = 0 \qquad (t \geq 0) \quad (-L/2 \leq x \leq +L/2),
\label{kdv_equation}
\end{equation}
where the solution $u = u(x,t)$ is subject to the periodic boundary condition
\begin{equation}
u(x+L, t) = u(x, t).
\label{ex5:eq:boundary_conditions}
\end{equation}
The equation is solved using the forward Euler and Crank-Nicholson method, and we conduct a stabiity analysis, showing that -- in this paricular case -- the choice of method is crucial.

\subsection{Analytical solution}

Let us solve the Korteweg-De Vries equation with separation of variables, writing one solution as
\begin{equation*}
	u_n(x,t) = X_n(x) \, T_n(t).
\end{equation*}
For the spatial part of the solution, let us use the periodic ansatz
\begin{equation*}
	X_n(x) = e^{i q_n x} \quad \text{with wavenumbers} \quad q_n = 2 \pi L / n.
\end{equation*}
Now insert $u_n(x,t) = X_n(x) \, T_n(t)$ into \cref{kdv_equation} and divide by $X_n(x) \, T_n(t)$ to get
\begin{equation*}
	\underbrace{\frac{\dot{T}_n(t)}{T_n(t)}}_{-i \omega_n} + \underbrace{\left( 1 + \pi^2 \right) \frac{X_n'(x)}{X_n(x)} + \frac{X_n'''(x)}{X_n(x)}}_{i \omega_n} = 0.
\end{equation*}
The first term is a function of $t$ only and the remaining terms are a function of $x$ only, so they must be constant.
In anticipation of the result, we label the constants $\mp i \omega_n$.
The temporal part gives
\begin{equation*}
	T_n(t) = e^{-i \omega_n t},
\end{equation*}
while inserting our ansatz $X_n(x) = e^{i q_n x}$ into the spatial part gives the dispersion relation
\begin{equation*}
	\omega_n = (1 + \pi^2) q_n - q_n^3.
\end{equation*}
The solution $u_n(x,t)$ is now fully specified.
Due to the linearity of \cref{kdv_equation} and the periodic boundary condition, we can superpose multiple solutions $u_n(x,t)$ into a general solution
\begin{equation}
	u(x,t) = \sum_{n=-\infty}^{+\infty} c_n \exp{\left( i(q_n x - \omega_n t) \right)},
	\label{fourier_series}
\end{equation}
which is a sum of plane waves propagating at different frequencies.

\subsection{Numerical solution method}

To find a numerical solution $U_m^n = U(x_m, t_n) \approx u(x_m, t_n) = u_m^n$ of the Korteweg-De Vries equation, we will discretize it with central differences in space and integrate over time with the Forward Euler method and the Crank-Nicholson method.
We will find the solution on the periodic spatial grid
\begin{figure}[htbp]
\begin{center}
\begin{tikzpicture}
\def\centerarc[#1](#2)(#3:#4:#5)% [draw options] (center) (initial angle:final angle:radius)
{ \draw[#1] ($(#2)+({#5*cos(#3)},{#5*sin(#3)})$) arc (#3:#4:#5); }
\centerarc[black,solid](0,0)(  0: 60:3);
\centerarc[black,dashed](0,0)( 60:150:3);
\centerarc[black,solid](0,0)(150:210:3);
\centerarc[black,dashed](0,0)(210:300:3);
\centerarc[black,solid](0,0)(300:360:3);
\node [fill=black, circle, scale=0.5, pin={  0:$x_0 = -L/2$}] at ({3*cos(  0)},{3*sin(  0)}) {};
\node [fill=black, circle, scale=0.5, pin={ 30:$x_1$}] at ({3*cos( 30)},{3*sin( 30)}) {};
\node [fill=black, circle, scale=0.5, pin={ 60:$x_2$}] at ({3*cos( 60)},{3*sin( 60)}) {};
\node [fill=black, circle, scale=0.5, pin={150:$x_{m-1}$}] at ({3*cos(150)},{3*sin(150)}) {};
\node [fill=black, circle, scale=0.5, pin={180:$x_m$}] at ({3*cos(180)},{3*sin(180)}) {};
\node [fill=black, circle, scale=0.5, pin={210:$x_{m+1}$}] at ({3*cos(210)},{3*sin(210)}) {};
\node [fill=black, circle, scale=0.5, pin={300:$x_{M-2}$}] at ({3*cos(300)},{3*sin(300)}) {};
\node [fill=black, circle, scale=0.5, pin={330:$x_{M-1}$}] at ({3*cos(330)},{3*sin(330)}) {};
\node [pin={[pin distance=0.05cm]15:$h$}] at ({3*cos( 15)},{3*sin( 15)}) {};
\node [pin={[pin distance=0.05cm]45:$h$}] at ({3*cos( 45)},{3*sin( 45)}) {};
\node [pin={[pin distance=0.05cm]165:$h$}] at ({3*cos(165)},{3*sin(165)}) {};
\node [pin={[pin distance=0.05cm]195:$h$}] at ({3*cos(195)},{3*sin(195)}) {};
\node [pin={[pin distance=0.05cm]315:$h$}] at ({3*cos(315)},{3*sin(315)}) {};
\node [pin={[pin distance=0.05cm]345:$h$}] at ({3*cos(345)},{3*sin(345)}) {};
\end{tikzpicture}
\end{center}
\label{ex4:fig:periodic_grid}
\caption{A depiction of the one dimensional periodic grid on which the linearized Korteweg-de Vries equation is computed. 
Since the boundary conditions -- equation \eqref{ex5:eq:boundary_conditions} -- are periodic, a circular grid is equivalent to an arbitrary number of successive linear grids.}
\end{figure}
of $M$ points.
For the first spatial derivative, we use the central difference
\begin{equation*}
	\dpd{u_m^n}{x} = \frac{u_{m+1}^n-u_{m-1}^n}{2 h} + \Oh(h^2).
\end{equation*}
We repeat the same finite difference three times to approximate the third order spatial derivative as 
\begin{equation*}
	\dpd[3]{u_m^n}{x} = \frac{u_{m+3}^n-3u_{m+1}^n+3u_{m-1}^n-u_{m-3}^n}{8 h^3} + \Oh(h^2).
\end{equation*}
Inserting these approximations into \cref{kdv_equation}, we get the intermediate result
\begin{equation*}
	\dpd{u_m^n}{t} = -\left(1+\pi^2\right) \frac{u_{m+1}^n-u_{m-1}^n}{2 h} - \frac{u_{m+3}^n-3u_{m+1}^n+3u_{m-1}^n-u_{m-3}^n}{8 h^3} + \Oh(h^2) \equiv F(u^n) + \Oh(h^2).
\end{equation*}
For later convenience, we write the Forward Euler method and Crank-Nicholson method collectively with the $\theta$-method.
This gives the final system of difference equations for the numerical solution
\begin{equation}\label{theta_method_discretized}
	\frac{U_m^{n+1} - U_m^n}{k} = (1-\theta) F(U^n) + \theta F(U^{n+1}),
\end{equation}
where the Forward Euler method or the Crank-Nicholson method is obtained by setting $\theta = 0$ or $\theta = 1/2$, respectively.
In matrix form, the system can be written
\begin{equation}
	\left(I - \theta k A\right) U^{n+1} = \left(I + \left(1-\theta\right) k A\right) U^n,
	\label{matrixeq}
\end{equation}
where $U^{n} = \begin{bmatrix} U_0^n & \dots & U_{M-1}^n \end{bmatrix}^T$ and $A = $
\newcommand\ca{\color{red}}
\newcommand\cb{\color{magenta}}
\newcommand\cc{\color{blue}}
\newcommand\cd{\color{cyan}}
% https://tex.stackexchange.com/a/571702
\begin{equation*}
\renewcommand{\arraystretch}{1.25} % stretch matrix vertically to make it square
\renewcommand{\arraycolsep}{4.15pt} % juuuust make matrix fill page width
% TODO: fix signs
\frac{-(1+\pi^2)}{2 h}
\begin{bmatrix}
0           & \ca +1   &             &             &             &             &             &             & \cb -1      \\
\cb -1      & 0        & \ca +1      &             &             &             &             &             &             \\
            & \cb -1   & 0           & \ca +1      &             &             &             &             &             \\
            &          & \cb -1      & 0           & \ca +1      &             &             &             &             \\
            &          &             & \cb \ddots  & \ddots      & \ca \ddots  &             &             &             \\
            &          &             &             & \cb -1      & 0           & \ca +1      &             &             \\
            &          &             &             &             & \cb -1      & 0           & \ca +1      &             \\
            &          &             &             &             &             & \cb -1      & 0           & \ca +1      \\
\ca +1      &          &             &             &             &             &             & \cb -1      & 0           \\
\end{bmatrix}
-\frac{1}{8 h^3}
\begin{bmatrix}
% 0      & +1/2h  &        &        & -1/2h  \\
% -1/2h  & 0      & +1/2h  &        &        \\
       % & \ddots & \ddots & \ddots &        \\
       % &        & -1/2h  & 0      & +1/2h  \\
% +1/2h  &        &        & -1/2h  & 0      \\
0           & \ca -3      & 0           & \cc +1      &             &             & \cd -1      & 0           & \cb +3      \\
\cb +3      & 0           & \ca -3      & 0           & \cc +1      &             &             & \cd -1      & 0           \\
0           & \cb +3      & 0           & \ca -3      & 0           & \cc +1      &             &             & \cd -1      \\
\cd -1      & 0           & \cb +3      & 0           & \ca -3      & 0           & \cc +1      &             &             \\
            & \cd \ddots  & \ddots      & \cb \ddots  & \ddots      & \ca \ddots  & \ddots      & \cc \ddots  &             \\
            &             & \cd -1      & 0           & \cb +3      & 0           & \ca -3      & 0           & \cc +1      \\
\cc +1      &             &             & \cd -1      & 0           & \cb +3      & 0           & \ca -3      & 0           \\
0           & \cc +1      &             &             & \cd -1      & 0           & \cb +3      & 0           & \ca -3      \\
\ca -3      & 0           & \cc +1      &             &             & \cd -1      & 0           & \cb +3      & 0           \\
\end{bmatrix}
.
\end{equation*}
where we have imposed periodic boundary conditions $U_m^n = U_{m+M}^n$ by simply wrapping the spatial derivative stencils around the matrix.
This is equivalent to calculating stencil indices modulo $M$, consistent with our circular grid.

We then solve the system by preparing $U^0$ from the initial condition $u(x, 0)$ and solve \cref{matrixeq} repeatedly to step forward in time.
Note that with constant steps $h$ and $k$ in both space and time, the matrices in \cref{matrixeq} are constant.
To save both memory and time, we represent them with sparse matrices. \cite{scipy_sparse}
In addition, to efficiently solve the same system with different right hand sides many times, we $LU$-factorize the sparse matrix for $I - \theta k A$. \cite{scipy_sparse_lu}
Note that with the Forward Euler method, $\theta = 0$ and this matrix reduces to the identity, so there is no system to solve -- the $U^{n+1}$ is given by simply multiplying the right side.

\begin{figure}[b]
\centering
\begin{tikzpicture}
\begin{groupplot}[
	group style={group size=5 by 1, horizontal sep=0.1cm, vertical sep=0.1cm},
	width=4.6cm, height=4.6cm,
	xmin=-1,xmax=+1,ymin=-1.1,ymax=+1.1,
	grid=major,xtick=\empty,ytick={0},yticklabels={,,},xticklabels={,,},
	% legend columns=-1, legend to name=thelegend, legend entries={exact, numerical},
]
	\pgfplotsinvokeforeach{0,1,...,4} {
		\ifthenelse{\equal{#1}{0}}{
		\nextgroupplot[ylabel={$u(x,t)$},xlabel=$x$,title={$t=#1/4$},
		legend columns=2, legend to name=thelegend, legend entries={exact solution, numerical solution}, legend style={column sep=2pt},
		];
		}{
		\nextgroupplot[xlabel=$x$,title={$t=#1/4$}];
		}
		\addplot [black,thick,domain=-1:+1,samples=100] {sin(deg(pi*(x-#1/4)))};
		\addplot [blue,thin] table [x expr=-1+2*\coordindex/800, y expr=\thisrowno{#1},skip first n=1] {exercise4/timeevol_sin.dat};
	};
\end{groupplot}
% \path (group c3r1.south east) -- node[below, inner ysep=30]{\pgfplotslegendfromname{thelegend}} (group c3r1.south west);
\node at ($(group c3r1) + (0,-2.5cm)$) {\pgfplotslegendfromname{thelegend}};
\end{tikzpicture}
%\caption{\label{sine_evolution}Time evolution of the numerical solution from the Crank-Nicholson method compared with the exact solution $\sin(\pi(x-t))$. The numerical solution is calculated using $M=800$ spatial grid points and $N=100$ temporal grid points.}
\caption{\label{sine_evolution}Comparison between the time evolution of the exact solution $u(x,t)=\sin(\pi(x-t))$ and the numerical solution from the Crank-Nicholson method with $h=1/799$ and $k=1/99$.}
\end{figure}

Next, we test our numerical solution on the problem defined by the initial condition $u(x, 0) = \sin(\pi x)$ on $x \in [-1, +1]$ with $L = 2$.
The analytical solution \ref{fourier_series} then gets nonzero contributions only from $n = \pm 1$, which gives the analytical solution $u(x, t) = \sin(\pi(x-t))$.
As shown in \cref{sine_evolution}, the solution represents a sine wave traveling with velocity $1$ to the right.

In \cref{ex4_euler_vs_crank_snapshots}, we compare snapshots of the numerical solution at $t = 1$ from the Forward Euler method and the Crank-Nicholson method.
Note that the Crank-Nicholson method approaches the exact solution with only $N=10$ time steps and under hundred spatial grid points $M$.
In contrast, the Forward Euler method seems to become unstable as the spatial resolution is increased, even with $N=100000$ time steps.

The convergence plot at $t=1$ in \cref{ex4_euler_vs_crank_convergence} supports our suspicions.
As we expect from the central finite differences, both methods show second order convergence in space for sufficiently refined grids. 
But the Forward Euler method diverges as $h$ decreases, although the divergence is delayed by also decreasing $k$.
The Crank-Nicholson method remains stable with much fewer time steps and much finer spatial grids.

\begin{figure}
\begin{tikzpicture}
\begin{axis}[ymin=-1.2, ymax=+1.2, title={Forward Euler ($N=100000$)},ylabel={$U(x,1)$},cycle list={
	{blue!100!black, solid, mark=square*, mark size=0.75pt},
	{blue!75!black, dashed, mark=square*, mark size=0.75pt},
	{blue!50!black, dotted, mark=square*, mark size=0.75pt, line width=0.5pt},
}]
\addplot [xlabel=$x$, color=black, domain=-1:+1, thick, samples=100, forget plot] {sin(deg(pi*(x-1)))} node [pos=0.5,pin={180:$u(x,1)$},inner sep=0pt] {};
\pgfplotsinvokeforeach{20,25,29} {
	\addplot table {exercise4/snapshot-forward-euler-M#1-N100000.dat};
	\addlegendentry{$M=#1$};
}
\end{axis}
\end{tikzpicture}
\begin{tikzpicture}
\begin{axis}[ymin=-1.2, ymax=+1.2, title={Crank-Nicholson ($N=10$)},yticklabels={,,},cycle list={
	{blue!100!black, style=solid, mark=square*, mark size=0.75pt},
	{blue!75!black, style=dashed, mark=square*, mark size=0.75pt},
	{blue!50!black, style=dotted, mark=square*, mark size=0.75pt},
}]
\addplot [xlabel=$x$, color=black, domain=-1:+1, thick, samples=100, forget plot] {sin(deg(pi*(x-1)))} node [pos=0.5,pin={180:$u(x,1)$},inner sep=0pt] {};
\pgfplotsinvokeforeach{25,50,75} {
	\addplot table {exercise4/snapshot-crank-nicholson-M#1-N10.dat};
	\addlegendentry{$M=#1$};
}
\end{axis}
\end{tikzpicture}
\caption{\label{ex4_euler_vs_crank_snapshots}
	Snapshots of the numerical solution $U(x,1)$ and the exact solution $u(x,1)$ for a constant number of time steps $N$, but varying number of grid points $M$ with the Forward Euler and Crank-Nicholson method.
	The left plot is meant to demonstrate the downfall of the Euler method and is not supposed to look pretty.
}
\end{figure}

\begin{figure}
\centering
\begin{tikzpicture}
\begin{loglogaxis}[xmin=4.5, xmax=46, ymin=1e-1, ymax=1e3,xlabel=$M$,ylabel={$\Ltwoerror{u-U}/\Ltwoerror{u}$},legend pos=north west,legend cell align=left,width=8.4cm,title=Forward Euler,xtick={5,10,15,20,25,30,35,40},xticklabels={5,10,15,20,25,30,35,40},cycle list={
	{cyan, mark=square*, mark size=1.5pt},
	{blue, mark=*, mark size=1.5pt},
	{teal, mark=triangle*, mark size=1.5pt},
}]
\addplot [black, dashed, samples=2, domain=1:100, forget plot] {280/x^2} node [pos=0.4, pin={45:$\Oh(h^2)$}] {};
\pgfplotsinvokeforeach{10000,100000,1000000} {
	\addplot table [x=M, y=E#1] {exercise4/convergence-forward-euler.dat};
	\addlegendentry{$N=#1$};
}
\end{loglogaxis}
\end{tikzpicture}
\begin{tikzpicture}
\begin{loglogaxis}[xlabel=$M$,legend pos=north east,legend cell align={left},width=8.4cm,title=Crank-Nicholson,xmin=5,xmax=2000,cycle list={
	{cyan, mark=square*, mark size=1.5pt},
	{teal, mark=*, mark size=1.5pt},
	{blue, mark=triangle*, mark size=1.5pt},
	{blue!30!black, mark=x, mark size=1.5pt},
	{blue!0!black, mark=square*, mark size=1.5pt},
}]
\addplot [dashed, domain=1:3000, samples=2, black, forget plot] {270/x^2} node [pos=0.32,pin={0:$\Oh(h^2)$},inner sep=0pt] {};
\pgfplotsinvokeforeach{10,100,1000,10000} {
	\addplot table [x=M, y=E#1] {exercise4/convergence-crank-nicholson.dat};
	\addlegendentry{$N=#1$};
}
\end{loglogaxis}
\end{tikzpicture}
\caption{\label{ex4_euler_vs_crank_convergence}Convergence plots with the discrete $L_2$ error of the numerical solution $U(x,t)$ for the Forward Euler and Crank-Nicholson method on the problem defined by the exact solution $u(x,t) = \sin(\pi(x-t))$.}
\end{figure}

\subsection{Stability analysis}

Motivated by the examples of the Euler method and the Crank-Nicholson method, we perform a Von Neumann analysis of their stability.
Just like the exact solution, the numerical solution is subject to periodic boundary conditions in space and can therefore be expanded in a Fourier series \cite{Kreyszig}
\begin{equation}
	U_m^n = U(x_m, t_n) = \sum_l C_l^n \exp \left(i q_l x_m\right).
\end{equation}
Consider now a single Fourier mode $C_l^n \exp (i q_l x_m)$ in this series.
Inserting it into \cref{theta_method_discretized}, dividing by $\exp(i q_l x_m)$ and expanding exponentials using Euler's identity $e^{ix} = \cos x + i \sin x$ gives
% TODO: need a cleaner version of the discretized equation
\begin{equation*}
% \frac{C_l^{n+1}-C_l^n}{k} = \left(\left(1-\theta\right)C_l^n+\theta C_l^{n+1}\right) \left(-(1+\pi^2) \frac{e^{i q_l h}-e^{-i q_l h}}{2h} - \frac{e^{3i q_l h}-e^{-3i q_l h}-3(e^{i q_l h}-e^{-i q_l h})}{8h^3}\right)
% \frac{C_l^{n+1}-C_l^n}{k} = i \left(\left(1-\theta\right)C_l^n+\theta C_l^{n+1}\right) \left(-(1+\pi^2) \frac{\sin(qh)}{h} - \frac{\sin^3(qh)}{h^3}\right)
\frac{C_l^{n+1}-C_l^n}{k} = i \left(\left(1-\theta\right)C_l^n+\theta C_l^{n+1}\right) f(q_l) , \quad \text{where} \,\, f(q_l) = \left(-\left(1+\pi^2\right) \frac{\sin(q_l h)}{h} - \frac{\sin^3(q_l h)}{h^3}\right).
\end{equation*}

Now look at the amplification factor $G_l = C_l^{n+1} / C_l^n$ of Fourier mode $l$ over one time step. 
With $\theta = 1/2$, the Crank-Nicholson method gives 
\begin{equation}
	G_l = \frac{1 + i k f(q_l) / 2}{1 - i k f(q_l) / 2} \quad \implies \quad \abs{G_l} = 1.
	\label{crank_nicholson_amplification_factor}
\end{equation}
The amplitude of all Fourier modes is thus preserved over time independently of $k$ and $h$, and we say the Crank-Nicholson method is \textbf{unconditionally stable}.

The Euler method has $\theta = 0$ and gives
\begin{equation}
	G_l = 1 + i k f(q_l) \quad \implies \quad \abs{G_l} = \sqrt{1 + k^2 f(q_l)^2}.
	\label{euler_amplification_factor}
\end{equation}
Since $\abs{\sin(q_l h)} \leq 1$ for all $q_l$, we can bound $f(q_l)$ by
\begin{equation*}
	\abs{f(q_l)} \leq \frac{(1+\pi^2)}{h} + \frac{1}{h^3} = \frac{1}{h^3} \left( (1+\pi^2) h^2 + 1 \right) \leq \frac{1}{h^3} \left( (1+\pi^2) L^2 + 1 \right).
\end{equation*}
Then $\abs{G_l} = \sqrt{1 + O(k^2 / h^6)} > 1$ for all $h$ and $k$, so each Fourier mode is amplified over time.
But the Von Neumann stability criterion $\abs{G_l} \leq 1 + O(k)$ \cite{owren} is still attained with $k \leq O(h^6)$, so the Forward Euler method is \textbf{conditionally stable}.
Only if $k/h^6 << 1$ does it remain stable, which explains the divergence for decreasing $h$ and fixed $k$ we found in \cref{ex4_euler_vs_crank_convergence} and why this is delayed by also decreasing $k$.

Thus, while the Euler method in theory is stable, it is unstable for practical combinations of $k$ and $h$.
The Crank-Nicholson method is far superior, as it remains stable over time and allows both smaller resolution in time and greater resolution in space.

\iffalse
\begin{figure}
\centering
% https://tex.stackexchange.com/questions/232690/color-coded-plots-with-colorbar-in-pgfplots
\begin{tikzpicture}
\begin{axis}[title={$N=10$},ylabel=$u$, xlabel=$x$, width=13cm, height=8cm, colorbar, colorbar style={ylabel=$M$}, xmin=-1, xmax=+1]
\addplot [color=black, domain=-1:+1, very thick] {sin(deg(pi*(x-1)))};
\addlegendentry{$\sin(\pi(x-1))$};
\pgfplotsinvokeforeach{20,40,60,80,100,200,300,400,500,600,700,800,900,1000} {
	\addplot [mesh, point meta=#1] table {exercise4/snapshot-crank-nicholson-M#1-N10.dat};
}
\end{axis}
\end{tikzpicture}
\begin{tikzpicture}
\begin{axis}[title={$N=100$},xlabel=$x$, width=13cm, height=8cm, colorbar, colorbar style={ylabel=$M$}, xmin=-1, xmax=+1]
\addplot [xlabel=$x$, color=black, domain=-1:+1, very thick] {sin(deg(pi*(x-1)))};
\addlegendentry{$\sin(\pi(x-1))$};
\pgfplotsinvokeforeach{20,40,60,80,100,200,300,400,500,600,700,800,900,1000} {
	\addplot [mesh, point meta=#1] table {exercise4/snapshot-crank-nicholson-M#1-N100.dat};
}
\end{axis}
\end{tikzpicture}
\end{figure}
\fi

\iffalse
\begin{figure}
\begin{tikzpicture}
\begin{axis}[
	height=10cm, width=1.00\textwidth, xlabel={$x$}, ylabel={$t$}, restrict z to domain=-5:+5, view={-10}{+10}, ymin=0, ymax=1,
]
	% \addplot3 [surf, mesh/cols=20] table {exercise4/cranknich.dat};
	\addplot3 [surf, mesh/cols=20] table {exercise4/crank-nicholson-M20-N400.dat};
\end{axis}
\end{tikzpicture}
\end{figure}
\fi

\subsection{Time evolution of norm}

% TODO: do proof quickly in Analytical solution instead?

The stability of the finite difference methods can be even better illustrated by investigating the time evolution of the $L_2$-norm of the solution.
To this end, we will first show that the $L_2$-norm of the analytical solution is preserved over time.
Then we will investigate the time evolution of the norm of numerical solutions.

The $L_2$-norm of the analytical solution is defined as
\begin{equation*}
	\Vert u(x,t) \Vert_2 = \left( \frac{1}{2} \int_{-L/2}^{+L/2} |u(x,t)|^2 \dif x \right)^{1/2}.
\end{equation*}
Now insert the solution \ref{fourier_series} and use orthogonality of the complex exponentials to get
\begin{equation*}
\int_{-L/2}^{+L/2} \dif x \left| u(x,t) \right|^2 = \sum_{m,n} c_m c_n^* \exp{(i(\omega_n-\omega_m)t)} \underbrace{\int_{-L/2}^{+L/2} \exp{\left(i(q_m-q_n)x\right)}}_{L \delta_{m n}} \dif x = L \sum_m \left| c_m \right|^2.
\end{equation*}
The final sum is independent of time, so the $L_2$-norm is indeed conserved.

We now investigate the norm of the numerical solution with the initial gaussian $u(x, 0) = \exp \left( -x^2 / 0.1 \right)$.
The time evolution illustrated in \cref{gaussian_evolution} shows how multiple modes are activated.
In \cref{norm_evolution}, we show how the norm of the numerical solution evolves over time.
The Euler method diverges even with tiny time steps, reflecting the amplification factor $G_l > 1$ found in \cref{euler_amplification_factor}.
In contrast, the Crank-Nicholson metohd is always stable and preserves the norm of the solution, reflecting the amplification factor $G_l = 1$ found in \cref{crank_nicholson_amplification_factor}.

\begin{figure}[t]
\centering
\begin{tikzpicture}
\begin{groupplot}[
	group style={group size=6 by 2, horizontal sep=0.1cm, vertical sep=0.1cm},
	width=4.1cm, height=4.1cm,
	xmin=-1,xmax=+1,ymin=-1.1,ymax=+1.1,
	grid=major,xtick=\empty,ytick={0},yticklabels={,,},xticklabels={,,},
]
	\pgfplotsinvokeforeach{0,1,...,11} {
		\ifthenelse{\equal{#1}{0}}{
		\nextgroupplot[ylabel={$u(x,t)$}];
		}{
			\ifthenelse{\equal{#1}{6}}{
				\nextgroupplot[ylabel={$u(x,t)$},xlabel=$x$];
			}{
				\ifthenelse{#1>6}{
					\nextgroupplot[xlabel=$x$];
				}{
					\nextgroupplot[];
				}
			}
		}
		\addplot [blue, thick] table [x expr=-1+2*\coordindex/800, y expr=\thisrowno{#1}] {exercise4/timeevol_exp.dat};
		\draw (0.0,-0.8) node {$t=#1/11$};
	};
\end{groupplot}
\end{tikzpicture}
\caption{\label{gaussian_evolution}Time evolution of a initial gaussian $u(x,0)=\exp(-x^2/0.1)$ computed from the Crank-Nicholson method on a grid with $M=800$ points in space and $N=100$ points in time.}
\end{figure}

\begin{figure}[h!]
\centering
\begin{tikzpicture}
\begin{groupplot}[
	group style={group size=2 by 1, horizontal sep=0.3cm},
	height=7cm,
	width=0.52\textwidth,
	xmin=-0.1, xmax=+1.1, ymin=0.2, ymax=1.2, restrict y to domain=0:10,
]
	\pgfplotscreateplotcyclelist{mycycle}{
		{black,dotted},
		{black,dashdotted},
		{black,solid},
		{blue,dotted},
		{blue,dashdotted},
		{blue,solid},
		{green!80!black,dotted},
		{green!80!black,dashdotted},
		{green!80!black,solid},
		{red,dotted},
		{red,dashdotted},
		{red,solid},
	};

	\nextgroupplot[title={Forward Euler (varying $N$)}, xlabel=$t$,ylabel={$\Vert U(t=1) \Vert_2$},cycle list name=mycycle,legend pos=north east];
	\addplot [forget plot, black, dashed, samples=2, domain=-1:2] {0.63} node [pos=0.31, pin={280:{$\Ltwoerror{u(x,t)}$}}] {};
	\addplot {0}; \addlegendentry{$N=30000$};
	\addplot {0}; \addlegendentry{$N=40000$};
	\addplot {0}; \addlegendentry{$N=50000$};
	\pgfplotsinvokeforeach{15,20,25} {
		\addplot table [x expr=\coordindex/100, y={forward-euler-M#1-N30000}] {exercise4/norm-evolution.dat};
		\addplot table [x expr=\coordindex/100, y={forward-euler-M#1-N40000}] {exercise4/norm-evolution.dat};
		\addplot table [x expr=\coordindex/100, y={forward-euler-M#1-N50000}] {exercise4/norm-evolution.dat};
	}

	\draw [blue ] (0.90, 0.75) node {$M=15$};
	\draw [green!80!black] (0.48, 0.95) node {$M=20$};
	\draw [red  ] (0.03, 0.89) node {$M=25$};

	\pgfplotscreateplotcyclelist{mycyclee}{
	};

	\nextgroupplot[title={Crank-Nicholson ($M=800$)}, xlabel=$t$,yticklabels={,,},cycle list={
		{red,thick},{blue,semithick},{green,thin}
	}, legend entries={$N=100$,$N=200$,$N=300$},
	];
	\addplot [forget plot, black, dashed, samples=2, domain=-1:2] {0.63} node [pos=0.31, pin={280:{$\Ltwoerror{u(x,t)}$}}] {};
	\pgfplotsinvokeforeach{800} {
		\addplot table [x expr=\coordindex/100, y={crank-nicholson-M#1-N100}] {exercise4/norm-evolution.dat};
		\addplot table [x expr=\coordindex/100, y={crank-nicholson-M#1-N200}] {exercise4/norm-evolution.dat};
		\addplot table [x expr=\coordindex/100, y={crank-nicholson-M#1-N300}] {exercise4/norm-evolution.dat};
	}
% \addplot [color=black] table [y={crank-nicholson-M800-N300}] {exercise4/norm-evolution.dat};
\end{groupplot}
\end{tikzpicture}
\caption{\label{norm_evolution}Time evolution of the discrete $L_2$-norm of the initial gaussian $u(x,0) = \exp(-x^2/0.1)$ computed from the Euler and the Crank-Nicholson method, on different grids.}
\end{figure}

The stability of the Crank-Nicholson method and the property that it preserves the amplitude of Fourier modes make it an optimal method for equations like the Korteweg-De Vries equation, where the analytical solution is known to have a constant norm.
