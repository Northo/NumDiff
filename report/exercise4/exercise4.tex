\section{Linearized Korteweg-De Vries equation in one dimension}

In this section, we will study the one-dimensional linearized Korteweg-De Vries equation
\begin{equation}
\pd{u}{t} + \left(1+\pi^2\right)\pd{u}{x} + \pd[3]{u}{x} = 0 \qquad (t \geq 0) \quad (-L/2 \leq x \leq +L/2),
\label{kdv_equation}
\end{equation}
where the solution $u = u(x,t)$ is subject to periodic boundary conditions
\begin{equation*}
u(x+L, t) = u(x, t).
\end{equation*}

\subsection{Analytical solution}

As the solution is periodic in space at every time $t$, it can be expressed as a Fourier series
\begin{equation}
u(x, t) = \sum_{n=-\infty}^{+\infty} c_n(t) \exp{(i q_n x)}
\label{fourier_series}
\end{equation}
with wavenumbers $q_n = 2 \pi n / L$ and time-dependent coefficients $c_n(t)$ that ensure spatial periodicity at all times.
This can be derived formally by separation of variables.
Inserting the Fourier series into \cref{kdv_equation} gives the condition
\begin{equation*}
	\sum_n \left( \dot{c}_n(t) + i \left( \left( 1+\pi^2 \right) q_n - q_n^3 \right) c_n(t) \right) \exp{(i q_n x)} = 0
\end{equation*}
on the coefficients.
Due to orthogonality of the Fourier basis functions $\exp(i q_n x)$, the sum can vanish only if all prefactors vanish separately.
This gives a first-order differential equation for each coefficient with the solution
\begin{equation}
	c_n(t) = c_n(0) \exp{\left( -i \left (\left(1+\pi^2\right)q_n - q_n^3\right) t \right) }.
	\label{fourier_coefficients}
\end{equation}

\subsection{Numerical solution method}

\iffalse
\begin{equation*}
	\dpd{u}{t} = F\left(\dpd{u}{x}, \dpd[3]{u}{x}\right), \qquad \text{where} \,\, F\left(\dpd{u}{x}, \dpd[3]{u}{x}\right) = -\left(1+\pi^2\right) \dpd{u}{x} - \dpd[3]{u}{x}
\end{equation*}
\fi

To find a numerical solution $U_m^n = U(x_m, t_n) \approx u(x_m, t_n) = u_m^n$ of the Korteweg-De Vries equation, we will discretize it with central differences in space and integrate over time with the Euler method and the Crank-Nicholson method.
For the first order spatial derivative, we use the central difference
\begin{equation*}
	\dpd{u_m^n}{x}    \approx \frac{\delta u_m^n}{2 h}       = \frac{u_{m+1}^n-u_{m-1}^n}{2 h}.
\end{equation*}
We repeat the same finite difference three times to approximate the third order spatial derivative as 
\begin{equation*}
	\dpd[3]{u_m^n}{x} \approx \frac{\delta^3 u_m^n}{(2 h)^3} = \frac{u_{m+3}^n-3u_{m+1}^n+3u_{m-1}^n-u_{m-3}^n}{8 h^3}.
\end{equation*}
Inserting these approximations into \cref{kdv_equation}, we get the intermediate result
\begin{equation*}
	\dpd{u_m^n}{t} \approx F(u^n) = -\left(1+\pi^2\right) \frac{u_{m+1}^n-u_{m-1}^n}{2 h} - \frac{u_{m+3}^n-3u_{m+1}^n+3u_{m-1}^n-u_{m-3}^n}{8 h^3}.
\end{equation*}
For later convenience, we write the Euler method and Crank-Nicholson method collectively with the $\theta$-method.
This gives the final system of difference equations for the numerical solution
\begin{equation}\label{theta_method_discretized}
	\frac{U_m^{n+1} - U_m^n}{k} = (1-\theta) F(U^n) + \theta F(U^{n+1}),
\end{equation}
where the Euler method or the Crank-Nicholson method is obtained by inserting $\theta = 0$ or $\theta = 1/2$, respectively.
In matrix form, the system can be written
\begin{equation}
	\left(I - \theta k A\right) U^{n+1} = \left(I - \left(1-\theta\right) k A\right) U^n,
	\label{matrixeq}
\end{equation}
where $U^{n} = \begin{bmatrix} U_0^n & \dots & U_{M-1}^n \end{bmatrix}^T$ and $A = $
\newcommand\ca{\color{red}}
\newcommand\cb{\color{magenta}}
\newcommand\cc{\color{blue}}
\newcommand\cd{\color{cyan}}
% https://tex.stackexchange.com/a/571702
\begin{equation*}
\renewcommand{\arraystretch}{1.25} % stretch matrix vertically to make it square
\renewcommand{\arraycolsep}{4.5pt} % juuuust make matrix fill page width
% TODO: fix signs
\frac{-1}{2 h}
\begin{bmatrix}
0           & \ca +1   &             &             &             &             &             &             & \cb -1      \\
\cb -1      & 0        & \ca +1      &             &             &             &             &             &             \\
            & \cb -1   & 0           & \ca +1      &             &             &             &             &             \\
            &          & \cb -1      & 0           & \ca +1      &             &             &             &             \\
            &          &             & \cb \ddots  & \ddots      & \ca \ddots  &             &             &             \\
            &          &             &             & \cb -1      & 0           & \ca +1      &             &             \\
            &          &             &             &             & \cb -1      & 0           & \ca +1      &             \\
            &          &             &             &             &             & \cb -1      & 0           & \ca +1      \\
\ca +1      &          &             &             &             &             &             & \cb -1      & 0           \\
\end{bmatrix}
-\frac{1+\pi^2}{h^3}
\begin{bmatrix}
% 0      & +1/2h  &        &        & -1/2h  \\
% -1/2h  & 0      & +1/2h  &        &        \\
       % & \ddots & \ddots & \ddots &        \\
       % &        & -1/2h  & 0      & +1/2h  \\
% +1/2h  &        &        & -1/2h  & 0      \\
0           & \ca -3      & 0           & \cc +1      &             &             & \cd -1      & 0           & \cb +3      \\
\cb +3      & 0           & \ca -3      & 0           & \cc +1      &             &             & \cd -1      & 0           \\
0           & \cb +3      & 0           & \ca -3      & 0           & \cc +1      &             &             & \cd -1      \\
\cd -1      & 0           & \cb +3      & 0           & \ca -3      & 0           & \cc +1      &             &             \\
            & \cd \ddots  & \ddots      & \cb \ddots  & \ddots      & \ca \ddots  & \ddots      & \cc \ddots  &             \\
            &             & \cd -1      & 0           & \cb +3      & 0           & \ca -3      & 0           & \cc +1      \\
\cc +1      &             &             & \cd -1      & 0           & \cb +3      & 0           & \ca -3      & 0           \\
0           & \cc +1      &             &             & \cd -1      & 0           & \cb +3      & 0           & \ca -3      \\
\ca -3      & 0           & \cc +1      &             &             & \cd -1      & 0           & \cb +3      & 0           \\
\end{bmatrix}
.
\end{equation*}
where we have imposed periodic boundary conditions $U_m^n = U_{m+M}^n$ by simply wrapping the spatial derivative stencils around the matrix.

We then solve the system by preparing $U^0$ from the initial condition $u(x, 0)$ and solve \cref{matrixeq} repeatedly to step forward in time.
Note that with the constant time step $k$, all matrices in \cref{matrixeq} are constant in time, and the process of solving the system many times can be accelerated by for example LU-factorizing the matrix on the left side.

\iffalse
\begin{figure}
\begin{tikzpicture}
\begin{axis}[
areaplot1/.style={fill opacity=0.50, fill=green, mark=none},
areaplot2/.style={fill opacity=0.50, fill=red, mark=none},
width=17cm,
height=12cm,
view={-28}{+30},
xlabel={$x$},
ylabel={$t$},
zmin=0,
ymajorgrids,
xmajorgrids,
zmajorgrids,
xtick distance=0.50,
ytick distance=0.25,
ztick distance=1,
legend cell align={left},
title={Crank-Nicholson ($M=800, \, N=100$)},
zticklabel={\pgfmathparse{\tick-1}$\pgfmathprintnumber{\pgfmathresult}$} % hack to offset labels by 1 (to make it possible to use filling, which is 0-based)
]
\pgfplotsinvokeforeach{5,4,...,1}{
	% TODO: set zlims correctly
	% Filled version
	% \addplot3 [areaplot1] table [x index=0,y expr={#1/5},z expr=\thisrowno{#1}+1] {exercise4/timeevol.dat} \closedcycle;
	% \pgfmathparse{int(round(5+#1))};
	% \pgfmathtruncatemacro\mymacro{round(5+#1)};
	% \addplot3 [areaplot2] table [x index=0,y expr={#1/5},z expr=\thisrowno{\mymacro}+1] {exercise4/timeevol.dat} \closedcycle;
	% \node[draw] at (0, 2) {\mymacro}; % for debug

	% Non-filled version
	\addplot3 [fill=gray, opacity=0.75, mark=none, draw=gray!80!black, thick] table [x index=0,y expr={(#1-1)/(5-1)},z expr=\thisrowno{#1}+1] {exercise4/timeevol.dat} \closedcycle;
	\ifthenelse{\equal{#1}{5}}{\addlegendentry{$u(x,t) = \sin(\pi(x-t))$}}{}
	\pgfmathparse{int(round(5+#1))};
	\pgfmathtruncatemacro\mymacro{round(5+#1)};
	\addplot3 [mark=none, color=red, thick] table [x index=0,y expr={(#1-1)/(5-1)},z expr=\thisrowno{\mymacro}+1] {exercise4/timeevol.dat};
	\ifthenelse{\equal{#1}{5}}{\addlegendentry{$U(x_m,t_n)$}}{}
	% \node[draw] at (0, 2) {\mymacro}; % for debug
}
\end{axis}
\end{tikzpicture}
\caption{\label{ex4_solution_3d}}
\end{figure}
\fi

\begin{figure}[b]
\centering
\begin{tikzpicture}
\begin{groupplot}[
	group style={group size=5 by 1, horizontal sep=0.1cm, vertical sep=0.1cm},
	width=4.6cm, height=4.6cm,
	xmin=-1,xmax=+1,ymin=-1.1,ymax=+1.1,
	grid=major,xtick=\empty,ytick={0},yticklabels={,,},xticklabels={,,},
	% legend columns=-1, legend to name=thelegend, legend entries={exact, numerical},
]
	\pgfplotsinvokeforeach{0,1,...,4} {
		\ifthenelse{\equal{#1}{0}}{
		\nextgroupplot[ylabel={$u(x,t)$},xlabel=$x$,title={$t=#1/4$},
		legend columns=2, legend to name=thelegend, legend entries={exact solution, numerical solution}, legend style={column sep=2pt},
		];
		}{
		\nextgroupplot[xlabel=$x$,title={$t=#1/4$}];
		}
		\addplot [black,thick,domain=-1:+1,samples=100] {sin(deg(pi*(x-#1/4)))};
		\addplot [blue,thin] table [x expr=-1+2*\coordindex/800, y expr=\thisrowno{#1},skip first n=1] {exercise4/timeevol_sin.dat};
	};
\end{groupplot}
% \path (group c3r1.south east) -- node[below, inner ysep=30]{\pgfplotslegendfromname{thelegend}} (group c3r1.south west);
\node at ($(group c3r1) + (0,-2.5cm)$) {\pgfplotslegendfromname{thelegend}};
\end{tikzpicture}
%\caption{\label{sine_evolution}Time evolution of the numerical solution from the Crank-Nicholson method compared with the exact solution $\sin(\pi(x-t))$. The numerical solution is calculated using $M=800$ spatial grid points and $N=100$ temporal grid points.}
\caption{\label{sine_evolution}Comparison between the time evolution of the exact solution $u(x,t)=\sin(\pi(x-t))$ and the numerical solution from the Crank-Nicholson method with $h=1/799$ and $k=1/99$.}
\end{figure}

Next, we test our numerical solution on the problem defined by the initial condition $u(x, 0) = \sin(\pi x)$ on $x \in [-1, +1]$ with $L = 2$.
The Fourier series of the analytical solution then has nonzero coefficients $c_{\pm1}(0) = \pm 1/2i$ and wavenumbers $q_{\pm 1} = \pm \pi$, which give rise to the analytical solution $u(x, t) = \sin(\pi(x-t))$ when inserted into \cref{fourier_series}.
As shown in \cref{sine_evolution}, the solution represents a sine wave traveling with velocity $1$ to the right.

In \cref{ex4_euler_vs_crank_snapshots}, we compare snapshots of the numerical solution at $t = 1$ from the Euler method and the Crank-Nicholson method.
Note that the Crank-Nicholson method approaches the exact solution with as little as $N=10$ time steps and a few hundred spatial grid points $M$, while the Euler method produces garbage with as many as $N=300 000$ time steps and becomes less stable as the number of spatial grid points increases.
The convergence plot of the error at $t=1$ in \cref{ex4_euler_vs_crank_convergence} supports our suspicions, showing the stable second order (TODO: why is it first order? is it supposed to be?) nature of the Crank-Nicholson method and the unstability of the Euler method.

\begin{figure}
\begin{tikzpicture}
\begin{axis}[ymin=-1.2, ymax=+1.2, title={Forward Euler ($N=300 000$)},ylabel={$U(x,1)$},cycle list={
	{blue!100!black, mark=square*, mark size=1pt},
	{blue!66!black, mark=square*, mark size=1pt},
	{blue!33!black, mark=square*, mark size=1pt},
}]
\pgfplotsinvokeforeach{5,15,25} {
	\addplot table {exercise4/snapshot-forward-euler-M#1-N300000.dat};
	\addlegendentry{$M=#1$};
}
\addplot [xlabel=$x$, color=black, domain=-1:+1, thick, samples=100] {sin(deg(pi*(x-1)))} node [pos=0.5,pin={0:$u(x,1)$},inner sep=0pt] {};
\end{axis}
\end{tikzpicture}
\begin{tikzpicture}
\begin{axis}[ymin=-1.2, ymax=+1.2, no markers,title={Crank-Nicholson ($N=10$)},yticklabels={,,},cycle list={
	{blue!100!black, mark=square*, mark size=1pt},
	{blue!66!black, mark=square*, mark size=1pt},
	{blue!33!black, mark=square*, mark size=1pt},
}]
\pgfplotsinvokeforeach{200,400,600} {
	\addplot table {exercise4/snapshot-crank-nicholson-M#1-N10.dat};
	\addlegendentry{$M=#1$};
}
\addplot [xlabel=$x$, color=black, domain=-1:+1, thick, samples=100] {sin(deg(pi*(x-1)))} node [pos=0.5,pin={180:$u(x,1)$},inner sep=0pt] {};
\end{axis}
\end{tikzpicture}
\caption{\label{ex4_euler_vs_crank_snapshots}
	Snapshots of the numerical solution $U(x,1)$ and the exact solution $u(x,1)$ for a constant number of time steps $N$, but varying number of grid points $M$ with the Forward Euler and Crank-Nicholson method.
	The left plot is meant to demonstrate the downfall of the Euler method and is not supposed to look pretty.
}
\end{figure}

\begin{figure}
\centering
\begin{tikzpicture}
\begin{loglogaxis}[xlabel=$M$,ylabel={$\Ltwoerror{u-U}/\Ltwoerror{u}$},legend pos=north west,width=8cm,title=Forward Euler,xtick=data,xticklabels={5,10,15,20,25,30},cycle list={
	{blue!100!black, mark=square*, mark size=1.5pt},
	{blue!66!black, mark=square*, mark size=1.5pt},
	{blue!33!black, mark=square*, mark size=1.5pt},
}]
\pgfplotsinvokeforeach{10000,20000,30000} {
	\addplot table [x=M, y=E#1] {exercise4/convergence-forward-euler.dat};
	\addlegendentry{$N=#1$};
}
\end{loglogaxis}
\end{tikzpicture}
\begin{tikzpicture}
\begin{loglogaxis}[xlabel=$M$,legend pos=north east,width=8cm,title=Crank-Nicholson,xmin=5,xmax=2000,cycle list={
	{blue!100!black, mark=square*, mark size=1.5pt},
	{blue!75!black, mark=square*, mark size=1.5pt},
	{blue!50!black, mark=square*, mark size=1.5pt},
	{blue!25!black, mark=square*, mark size=1.5pt},
	{blue!0!black, mark=square*, mark size=1.5pt},
}]
\addplot [dashed, domain=1:3000, samples=2, black, forget plot] {40/x^1} node [pos=0.4,pin={0:$\Oh(h^1)$},inner sep=0pt] {};
\pgfplotsinvokeforeach{10,20,30,40,50} {
	\addplot table [x=M, y=E#1] {exercise4/convergence-crank-nicholson.dat};
	\addlegendentry{$N=#1$};
}
\end{loglogaxis}
\end{tikzpicture}
\caption{\label{ex4_euler_vs_crank_convergence}Convergence plots with the discrete $L_2$ error of the numerical solution $U(x,t)$ for the Forward Euler and Crank-Nicholson method on the problem defined by the exact solution $u(x,t) = \sin(\pi(x-t))$.}
\end{figure}

\subsection{Stability analysis}

Motivated by the examples of the Euler method and the Crank-Nicholson method, we perform a Von Neumann analysis of their stability.
Just like the exact solution, the numerical solution is subject to periodic boundary conditions in space and can therefore be expanded in a Fourier series
\begin{equation}
	U_m^n = U(x_m, t_n) = \sum_l C_l^n \exp \left(i q_l x_m\right).
\end{equation}
Consider now a single Fourier mode $C_l^n \exp (i q_l x_m)$ in this series.
Inserting it into \cref{theta_method_discretized}, dividing by $\exp(i q_l x_m)$ and expanding exponentials using Euler's identity gives
% TODO: need a cleaner version of the discretized equation
\begin{equation*}
% \frac{C_l^{n+1}-C_l^n}{k} = \left(\left(1-\theta\right)C_l^n+\theta C_l^{n+1}\right) \left(-(1+\pi^2) \frac{e^{i q_l h}-e^{-i q_l h}}{2h} - \frac{e^{3i q_l h}-e^{-3i q_l h}-3(e^{i q_l h}-e^{-i q_l h})}{8h^3}\right)
% \frac{C_l^{n+1}-C_l^n}{k} = i \left(\left(1-\theta\right)C_l^n+\theta C_l^{n+1}\right) \left(-(1+\pi^2) \frac{\sin(qh)}{h} - \frac{\sin^3(qh)}{h^3}\right)
\frac{C_l^{n+1}-C_l^n}{k} = i \left(\left(1-\theta\right)C_l^n+\theta C_l^{n+1}\right) f(q_l) , \quad \text{where} \,\, f(q_l) = \left(-\left(1+\pi^2\right) \frac{\sin(q_l h)}{h} - \frac{\sin^3(q_l h)}{h^3}\right).
\end{equation*}

Now look at the amplification factor $G_l = C_l^{n+1} / C_l^n$ of Fourier mode $l$ over one time step. 
With $\theta = 1/2$, the Crank-Nicholson method gives 
\begin{equation}
	G_l = \frac{1 + i k f(x) / 2}{1 - i k f(x) / 2} \quad \implies \quad \abs{G_l} = 1.
	\label{crank_nicholson_amplification_factor}
\end{equation}
The amplitude of all Fourier modes is thus preserved over time independently of $k$ and $h$, and we say the Crank-Nicholson method is \textbf{unconditionally stable}.

The Euler method has $\theta = 0$ and gives
\begin{equation}
	G_l = 1 + i k f(x) \quad \implies \quad \abs{G_l} = \sqrt{1 + k^2 f(q_l)^2}.
	\label{euler_amplification_factor}
\end{equation}
Since $\abs{\sin(q_l h)} \leq 1$ for all $q_l$, we can bound $f(q_l)$ by
\begin{equation*}
	\abs{f(q_l)} \leq \frac{(1+\pi^2)}{h} + \frac{1}{h^3} = \frac{1}{h^3} \left( (1+\pi^2) h^2 + 1 \right) \leq \frac{1}{h^3} \left( (1+\pi^2) L^2 + 1 \right).
\end{equation*}
Then $\abs{G_l} = \sqrt{1 + O(k^2 / h^6)}$.
% TODO: cite von neumann stability criterion
The Von Neumann stability criterion $\abs{G_l} \leq 1 + O(k)$ \cite{owren} is attained only with $k \leq O(h^6)$, corresponding to \emph{extremely small} time steps.
Thus, while the Euler method in theory is \textbf{conditionally stable}, it is unstable for all practical time steps.
The Crank-Nicholson method is far superior, as it remains stable while allowing much coarser spatial resolution and larger time steps.

% TODO: one figure with Euler, small M large N
% TODO: one figure with CN, 

\iffalse
\begin{figure}
\centering
% https://tex.stackexchange.com/questions/232690/color-coded-plots-with-colorbar-in-pgfplots
\begin{tikzpicture}
\begin{axis}[title={$N=10$},ylabel=$u$, xlabel=$x$, width=13cm, height=8cm, colorbar, colorbar style={ylabel=$M$}, xmin=-1, xmax=+1]
\addplot [color=black, domain=-1:+1, very thick] {sin(deg(pi*(x-1)))};
\addlegendentry{$\sin(\pi(x-1))$};
\pgfplotsinvokeforeach{20,40,60,80,100,200,300,400,500,600,700,800,900,1000} {
	\addplot [mesh, point meta=#1] table {exercise4/snapshot-crank-nicholson-M#1-N10.dat};
}
\end{axis}
\end{tikzpicture}
\begin{tikzpicture}
\begin{axis}[title={$N=100$},xlabel=$x$, width=13cm, height=8cm, colorbar, colorbar style={ylabel=$M$}, xmin=-1, xmax=+1]
\addplot [xlabel=$x$, color=black, domain=-1:+1, very thick] {sin(deg(pi*(x-1)))};
\addlegendentry{$\sin(\pi(x-1))$};
\pgfplotsinvokeforeach{20,40,60,80,100,200,300,400,500,600,700,800,900,1000} {
	\addplot [mesh, point meta=#1] table {exercise4/snapshot-crank-nicholson-M#1-N100.dat};
}
\end{axis}
\end{tikzpicture}
\end{figure}
\fi

\iffalse
\begin{figure}
\begin{tikzpicture}
\begin{axis}[
	height=10cm, width=1.00\textwidth, xlabel={$x$}, ylabel={$t$}, restrict z to domain=-5:+5, view={-10}{+10}, ymin=0, ymax=1,
]
	% \addplot3 [surf, mesh/cols=20] table {exercise4/cranknich.dat};
	\addplot3 [surf, mesh/cols=20] table {exercise4/crank-nicholson-M20-N400.dat};
\end{axis}
\end{tikzpicture}
\end{figure}
\fi

\subsection{Time evolution of norm}

% TODO: do proof quickly in Analytical solution instead?

The stability of the finite difference methods can be even better illustrated by investigating the time evolution of the $L_2$-norm of the solution.
To this end, we will first show that the $L_2$-norm of the analytical solution is preserved over time.
Then we will show the time evolution of the $L_2$-norm of numerical solutions.

The $L_2$-norm of the analytical solution is defined as
\begin{equation*}
	\Vert u(x,t) \Vert_2 = \left( \frac{1}{2} \int_{-L/2}^{+L/2} |u(x,t)|^2 \dif x \right)^{1/2}.
\end{equation*}
To understand why it is constant in time, insert the Fourier expansion \cref{fourier_series} for $u(x,t)$ to get
\begin{equation*}
\int_{-1}^{+1} \dif x \left| u(x,t) \right|^2 = \sum_{m,n} c_m(t) c_n^*(t) \underbrace{\int_{-L/2}^{+L/2} \exp{\left(i(q_m-q_n)x\right)}}_{L \delta_{m n}} \dif x = L \sum_n \left| c_n(t) \right|^2.
\end{equation*}
Since the Fourier coefficients \eqref{fourier_coefficients} have constant magnitude $|c_n(t)| = |c_n(0)|$, the $L_2$-norm is the same at any time $t$.
Note that it is the \emph{odd} number of spatial derivatives in \cref{kdv_equation} that gives the exponential in \cref{fourier_coefficients} a purely imaginary argument, and preserves the amplitude of each Fourier mode.

We now investigate the norm of the numerical solution with the initial gaussian $u(x, 0) = \exp \left( -x^2 / 0.1 \right)$.
The time evolution illustrated in \cref{gaussian_evolution} shows how multiple Fourier modes are activated.
In \cref{norm_evolution}, we show how the norm of the numerical solution evolves over time.
The Euler method diverges even with tiny time steps, reflecting the amplification factor $G_l > 1$ found in \cref{euler_amplification_factor}.
In contrast, the Crank-Nicholson metohd is always stable and preserves the norm of the solution, reflecting the amplification factor $G_l = 1$ found in \cref{crank_nicholson_amplification_factor}.

\begin{figure}[t]
\centering
\begin{tikzpicture}
\begin{groupplot}[
	group style={group size=6 by 2, horizontal sep=0.1cm, vertical sep=0.1cm},
	width=4.1cm, height=4.1cm,
	xmin=-1,xmax=+1,ymin=-1.1,ymax=+1.1,
	grid=major,xtick=\empty,ytick={0},yticklabels={,,},xticklabels={,,},
]
	\pgfplotsinvokeforeach{0,1,...,11} {
		\ifthenelse{\equal{#1}{0}}{
		\nextgroupplot[ylabel={$u(x,t)$}];
		}{
			\ifthenelse{\equal{#1}{6}}{
				\nextgroupplot[ylabel={$u(x,t)$},xlabel=$x$];
			}{
				\ifthenelse{#1>6}{
					\nextgroupplot[xlabel=$x$];
				}{
					\nextgroupplot[];
				}
			}
		}
		\addplot [blue, thick] table [x expr=-1+2*\coordindex/800, y expr=\thisrowno{#1}] {exercise4/timeevol_exp.dat};
		\draw (0.0,-0.8) node {$t=#1/11$};
	};
\end{groupplot}
\end{tikzpicture}
\caption{\label{gaussian_evolution}Time evolution of a initial gaussian $u(x,0)=\exp(-x^2/0.1)$ computed from the Crank-Nicholson method on a grid with $M=800$ points in space and $N=100$ points in time.}
\end{figure}

\begin{figure}[h!]
\centering
\begin{tikzpicture}
\begin{groupplot}[
	group style={group size=2 by 1, horizontal sep=0.3cm},
	height=7cm,
	width=0.52\textwidth,
	ymin=0.4, ymax=0.5, restrict y to domain=0:10,
]
	\pgfplotscreateplotcyclelist{mycycle}{
		{black,dotted},
		{black,dashed},
		{black,solid},
		{blue!100!black,dotted},
		{blue!80!black,dashed},
		{blue!60!black,solid},
		{green!100!black,dotted},
		{green!80!black,dashed},
		{green!60!black,solid},
		{red!100!black,dotted},
		{red!80!black,dashed},
		{red!60!black,solid},
	};

	\nextgroupplot[title={Forward Euler}, xlabel=$t$,ylabel={$\Vert U(t=1) \Vert_2$},cycle list name=mycycle,legend pos=north east,restrict y to domain=0:1.0];
	\addplot {0}; \addlegendentry{$N=30000$};
	\addplot {0}; \addlegendentry{$N=40000$};
	\addplot {0}; \addlegendentry{$N=50000$};
	\pgfplotsinvokeforeach{15,20,25} {
		\addplot table [x expr=\coordindex/100, y={forward-euler-M#1-N30000}] {exercise4/norm-evolution.dat};
		\addplot table [x expr=\coordindex/100, y={forward-euler-M#1-N40000}] {exercise4/norm-evolution.dat};
		\addplot table [x expr=\coordindex/100, y={forward-euler-M#1-N50000}] {exercise4/norm-evolution.dat};
	}

	\draw [blue ] (0.80, 0.43) node {$M=15$};
	\draw [green] (0.70, 0.46) node {$M=20$};
	\draw [red  ] (0.03, 0.49) node {$M=25$};

	\pgfplotscreateplotcyclelist{mycyclee}{
	};

	\nextgroupplot[title={Crank-Nicholson ($M=800$)}, xlabel=$t$,yticklabels={,,},cycle list={
		{red,thick},{blue,semithick},{green,thin}
	}, legend entries={$N=100$,$N=200$,$N=300$},
	];
	\pgfplotsinvokeforeach{800} {
		\addplot table [x expr=\coordindex/100, y={crank-nicholson-M#1-N100}] {exercise4/norm-evolution.dat};
		\addplot table [x expr=\coordindex/100, y={crank-nicholson-M#1-N200}] {exercise4/norm-evolution.dat};
		\addplot table [x expr=\coordindex/100, y={crank-nicholson-M#1-N300}] {exercise4/norm-evolution.dat};
	}
% \addplot [color=black] table [y={crank-nicholson-M800-N300}] {exercise4/norm-evolution.dat};
\end{groupplot}
\end{tikzpicture}
\caption{\label{norm_evolution}Time evolution of the discrete $L_2$-norm of the initial gaussian $u(x,0) = \exp(-x^2/0.1)$ computed from the Euler and the Crank-Nicholson method, on different grids.}
\end{figure}

The stability and norm preservation of the Crank-Nicholson method makes it the method of choice for problems like this, where the analytical solution is known to have the same property.
