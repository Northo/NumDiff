\section{Heat equation in one dimension}
In this seection, we will solve one-dimensional differential equations with time evolution. 
Firstly, we will solve the one-dimensional heat equation with Neumann boundary conditions, using both the Backward Euler method and Crank-Nicolson's method.
We will then analyse the error of our methods using a manufactured solution to which the analytical solution is known.
For this, we will use both uniform and adaptive mesh refinement, and the different methods will be discussed.
Finally, we will solve the inviscid Burgers' equation and evaluate the breaking which this equation exhibits.

Firstly, we consider the one-dimensional heat equation with Neumann boundary conditions
\begin{equation*}
    u_t = u_{xx}, \quad u_x(0,t) = u_x(1,t) = 0, \quad u(x,0) = 2\pi x - \sin(2\pi x),
\end{equation*}
valid for $x \in [0,1] := \Omega$ and $t > 0$.
The equation can be solved numeriacally by dividing the interval $\Omega$ in $M$ parts, as described in section \ref{task_1}.
We also divide the time interval $t \in [0,t_{end}]$ into parts, so that we approximate the solution at $N$ finite times $t_n = nk, \quad n = 1, \ldots N$.
For $t = 0$, the solution is the given initial condition.


\begin{figure}[h]
    \centering
    \begin{tikzpicture}
    \begin{loglogaxis}[
        title={2a Convergence plot L2 discrete rel. error},
        xlabel={$M$},
        ylabel={Relative error},
        ymax=10,
        ]
        \addplot[color=blue,mark=x] table[x={M}, y={err}] {exercise2/data_ka/2a_BE_discrete_err_N100_Mref1000_tend1.dat};
        \addplot[color=red,mark=*] table[x={M}, y={err}] {exercise2/data_ka/2a_CN_discrete_err_N100_Mref1000_tend1.dat};
        \legend{Backward Euler, Crank-Nicolson};
    \end{loglogaxis}
\end{tikzpicture}

\end{figure}

\begin{figure}[h]
    \centering
    \begin{tikzpicture}
    \begin{loglogaxis}[
        title={2a Convergence plot, l2 cont. rel. err.},
        xlabel={$M$},
        ylabel={Relative error},
        ]
        \addplot[color=blue,mark=x] table[x={M}, y={err}] {exercise2/data_ka/2a_BE_continous_err_N100_Mref1000_tend1.dat};
        \addplot[color=red,mark=*] table[x={M}, y={err}] {exercise2/data_ka/2a_CN_continous_err_N100_Mref1000_tend1.dat};
        \legend{Backward Euler, Crank-Nicolson};
    \end{loglogaxis}
\end{tikzpicture}

\end{figure}

\begin{figure}[h]
    \centering
    \begin{tikzpicture}
    \begin{loglogaxis}[
        title={2b Convergence plot L2 discrete rel err},
        xlabel={$M$},
        ylabel={Relative error},
        legend pos=south west,
        ]
        \addplot[color=blue,mark=x] table[x={M}, y={err}] {exercise2/data_ka/2b_UMR_BE_discrete_err_N1000_tend1.dat};
        \addplot[color=red,mark=*] table[x={M}, y={err}] {exercise2/data_ka/2b_UMR_CN_discrete_err_N1000_tend1.dat};
        \addplot[color=black,mark=diamond] table[x={M}, y={err}] {exercise2/data_ka/2b_UMR_BE_discrete_err_N10000_tend1.dat};
        \addplot[color=gray,mark=triangle] table[x={M}, y={err}] {exercise2/data_ka/2b_UMR_CN_discrete_err_N10000_tend1.dat};


        \legend{Backward Euler N=1000, Crank-Nicolson N=1000, Backward Euler N=10000, Crank-Nicolson N=10000};
    \end{loglogaxis}
\end{tikzpicture}

\end{figure}

\begin{figure}[h]
    \centering
    \begin{tikzpicture}
    \begin{loglogaxis}[
        title={2b Convergence plot l2 continous rel. err.},
        xlabel={$M$},
        ylabel={Relative error},
        legend pos=south west,
        ]
        \addplot[color=blue,mark=x] table[x={M}, y={err}] {exercise2/data_ka/2b_UMR_BE_continous_err_N1000_tend1.dat};
        \addplot[color=red,mark=*] table[x={M}, y={err}] {exercise2/data_ka/2b_UMR_CN_continous_err_N1000_tend1.dat};
        \addplot[color=black,mark=diamond] table[x={M}, y={err}] {exercise2/data_ka/2b_UMR_BE_continous_err_N10000_tend1.dat};
        \addplot[color=gray,mark=triangle] table[x={M}, y={err}] {exercise2/data_ka/2b_UMR_CN_continous_err_N10000_tend1.dat};
        \legend{Backward Euler N=1000, Crank-Nicolson N=1000, Backward Euler N=10000, Crank-Nicolson N=10000};
    \end{loglogaxis}
\end{tikzpicture}

\end{figure}

\textbf{AMR - data likely to change:}

\begin{figure}[h]
    \centering
    \begin{tikzpicture}
    \begin{loglogaxis}[
        title={2b AMR Convergence plot L2 discrete rel err},
        xlabel={$M$},
        ylabel={Relative error},
        ]
        \addplot[color=blue,mark=x] table[x={M}, y={err}] {exercise2/data_ka/2b_AMR_BE_discrete_err_N1000_tend1.dat};
        \addplot[color=red,mark=*] table[x={M}, y={err}] {exercise2/data_ka/2b_AMR_CN_discrete_err_N1000_tend1.dat};
        \legend{Backward Euler, Crank-Nicolson};
    \end{loglogaxis}
\end{tikzpicture}

    \caption{AMR}
\end{figure}

\begin{figure}[h]
    \centering
    \begin{tikzpicture}
    \begin{loglogaxis}[
        title={2b AMR Convergence plot l2 continous rel. err.},
        xlabel={$M$},
        ylabel={Relative error},
        ]
        \addplot[color=blue,mark=x] table[x={M}, y={err}] {exercise2/data_ka/2b_AMR_BE_continous_err_N1000_tend1.dat};
        \addplot[color=red,mark=*] table[x={M}, y={err}] {exercise2/data_ka/2b_AMR_CN_continous_err_N1000_tend1.dat};
        \legend{Backward Euler, Crank-Nicolson};
    \end{loglogaxis}
\end{tikzpicture}

    \caption{AMR}
\end{figure}
