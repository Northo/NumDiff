\section{Heat equation in one dimension}
In this section, we consider the one-dimensional heat equation for $u = u(x, t)$, 
\begin{equation*}
    u_t = u_{xx}, \quad u(x, 0) = f(x), \quad x \in [0,1] := \Omega, 
    \label{eq:heat-eq}
\end{equation*}
with either Neumann or Dirchelet boundaryconditions, 
and solve it numerically using both the Backward Euler method and Crank-Nicolson method. 
These are $\mathcal{O}(k+h^2)$ and $\mathcal{O}(k^2+h^2)$ methods respectively, 
and we will analyze and compare the convergence of the two methods using mesh refinement as we did in section \ref{task_1}. 

\subsection{Numerical solution method}
To solve the heat equation numerically we first perform semi-discretization, 
i.e. we do spatial discretization and keep the time continous. 
The interval $\Omega$ is divided into $M$ equidistant points with sepparation $h=1/(M-1)$, 
as described in section \ref{task_1}, 
and we approximate the spatial derivative with central finite differences. 
Introducing $v_m(t), m = 0, ..., M$ as approximations to $u(x_m, t)$, 
the result is now that we have gone from a PDE to a system of ODEs, 
\begin{equation*}
    \frac{dv_m(t)}{dt} = \frac{1}{h^2} \delta_x^2 v_m(t), 
    \quad v_m(0) = f(x_m). 
\end{equation*}

The problem is then generally solved by imposing the boundary conditions, 
and numerically integrating the equations in time, 
using e.g. Euler's method. 
For convenience we employ the $\theta$-method, 
\begin{equation*}
    General theta method, 
\end{equation*}
and obtain 
\begin{equation}
    theta method for heat equation.
    \label{eq:theta-heat}
\end{equation}
Integrating with constant time step $k$ results in a uniform grid for the time interval as well. 
As for the spatial grid, we divide the time interval into $N$ equidistant points, 
so that we approximate the solution at $N$ finite times $t_n = nk, \quad n = 1, \ldots N$, 
\quad $k = 1/(N-1)$. 

\begin{figure}[ht]
    \centering
    \begin{tikzpicture}
%% X-axis, courtesy of task force 1:
\draw (0,0) -- (3,0);
\draw[dashed] (3,0) -- (6,0);
\draw (6,0) -- (9,0);
\filldraw (0.0,0) circle (2pt) node[anchor=north] {$x_0 = 0$};
\filldraw (1.5,0) circle (2pt) node[anchor=north] {$x_1$};
\filldraw (3.0,0) circle (2pt) node[anchor=north] {$x_2$};
\filldraw (4.5,0) circle (2pt) node[anchor=north] {$x_m$};
\filldraw (6.0,0) circle (2pt) node[anchor=north] {$x_{M-1}$};
\filldraw (7.5,0) circle (2pt) node[anchor=north] {$x_{M}$};
\filldraw (9.0,0) circle (2pt) node[anchor=north] {$x_{M+1} = 1$};
\node[anchor=south] at (0.75, 0) {$h$};
\node[anchor=south] at (2.25, 0) {$h$};
\node[anchor=south] at (6.75, 0) {$h$};
\node[anchor=south] at (8.25, 0) {$h$};

% Y-axis
\draw (0,0) -- (0,3);
\draw[dashed] (0,3) -- (0,6);
\draw (0,6) -- (0,9);
%\filldraw (0.0,0) circle (2pt) node[anchor=west] {$x_0 = 0$};
\filldraw (0,1.5) circle (2pt) node[anchor=east] {$t_1$};
\filldraw (0,3) circle (2pt) node[anchor=east] {$t_2$};
\filldraw (0,4.5) circle (2pt) node[anchor=east] {$t_n$};
\filldraw (0,6) circle (2pt) node[anchor=east] {$t_{N-2}$};
\filldraw (0,7.5) circle (2pt) node[anchor=east] {$t_{N-1}$};
\filldraw (0,9) circle (2pt) node[anchor=east] {$t_{N} = t_{end}$};

% Y-axis on right side
\draw (9.0,0) -- (9.0,3);
\draw[dashed] (9.0,3) -- (9.0,6);
\draw (9.0,6) -- (9.0,9.0);
% Labels on the right side might be unneessary.
\filldraw (9.0,1.5) circle (2pt) node[anchor=west] {$t_1$};
\filldraw (9.0,3.0) circle (2pt) node[anchor=west] {$t_2$};
\filldraw (9.0,4.5) circle (2pt) node[anchor=west] {$t_n$};
\filldraw (9.0,6.0) circle (2pt) node[anchor=west] {$t_{N-2}$};
\filldraw (9.0,7.5) circle (2pt) node[anchor=west] {$t_{N-1}$};
\filldraw (9.0,9.0) circle (2pt) node[anchor=west] {$t_{N} = t_{end}$};

% Grid
\draw[dotted] (1.5,0) -- (1.5,9.0);
\draw[dotted] (3.0,0) -- (3.0,9.0);
\draw[dotted] (4.5,0) -- (4.5,9.0);
\draw[dotted] (6.0,0) -- (6.0,9.0);
\draw[dotted] (7.5,0) -- (7.5,9.0);
\draw[dotted] (9.0,0) -- (9.0,9.0);

\draw[dotted] (0,1.5) -- (9.0,1.5);
\draw[dotted] (0,3.0) -- (9.0,3.0);
\draw[dotted] (0,4.5) -- (9.0,4.5);
\draw[dotted] (0,6.0) -- (9.0,6.0);
\draw[dotted] (0,7.5) -- (9.0,7.5);
\draw[dotted] (0,9.0) -- (9.0,9.0);

% Unknown nodes in the grid
\draw (1.5,1.5) circle (2pt);
\draw (1.5,3.0) circle (2pt);
\draw (1.5,4.5) circle (2pt);
\draw (1.5,6.0) circle (2pt);
\draw (1.5,7.5) circle (2pt);
\draw (1.5,9.0) circle (2pt);
\draw (3.0,1.5) circle (2pt);
\draw (3.0,3.0) circle (2pt);
\draw (3.0,4.5) circle (2pt);
\draw (3.0,6.0) circle (2pt);
\draw (3.0,7.5) circle (2pt);
\draw (3.0,9.0) circle (2pt);
\draw (4.5,1.5) circle (2pt);
\draw (4.5,3.0) circle (2pt);
\draw (4.5,4.5) circle (2pt);
\draw (4.5,6.0) circle (2pt);
\draw (4.5,7.5) circle (2pt);
\draw (4.5,9.0) circle (2pt);
\draw (6.0,1.5) circle (2pt);
\draw (6.0,3.0) circle (2pt);
\draw (6.0,4.5) circle (2pt);
\draw (6.0,6.0) circle (2pt);
\draw (6.0,7.5) circle (2pt);
\draw (6.0,9.0) circle (2pt);
\draw (7.5,1.5) circle (2pt);
\draw (7.5,3.0) circle (2pt);
\draw (7.5,4.5) circle (2pt);
\draw (7.5,6.0) circle (2pt);
\draw (7.5,7.5) circle (2pt);
\draw (7.5,9.0) circle (2pt);
\draw (9.0,1.5) circle (2pt);
\draw (9.0,3.0) circle (2pt);
\draw (9.0,4.5) circle (2pt);
\draw (9.0,6.0) circle (2pt);
\draw (9.0,7.5) circle (2pt);
\draw (9.0,9.0) circle (2pt);

\end{tikzpicture}

\end{figure}

Setting the value of $\theta$ in \eqref{eq:theta-heat} determines the specific numerical scheme, 
we have
\begin{equation*}
\begin{split}
    \text{Forward Euler} \quad \theta = 0 \\
    \text{Backward Euler} \quad \theta = 1 \\
    \text{Crank-Nicolson} \quad \theta = \frac{1}{2}, \\
\end{split}
\end{equation*}
and we will as mentioned consider the Bacward-Euler and Crank-Nicolson methods. 

Imposing boundary conditions and some matrix equations here I guess.

We start by considering the heat equation with the following Neumann boundary conditions and initial condition, 
\begin{equation}
    u_x(0,t) = u_x(1,t) = 0, \quad u(x,0) = 2\pi x - \sin(2\pi x),
    \label{eq:2a}
\end{equation}

\begin{figure}[ht]
    \centering
    \begin{tikzpicture}
    \begin{axis}
    [
        xlabel={$t$},
        ylabel={$x$},
        zlabel={$U$},
        view={30}{20},
        height=6.5cm,
        mesh/ordering=y varies,
        grid,
    ]
    \addplot3[
      surf,
      mesh/cols=50,
      shader=faceted,
      opacity=0.9,
         ] table[x={t}, y={x}, z={U}]
             {exercise2/data_ka/2a_surface.dat};
    \end{axis}
\end{tikzpicture}

    \caption{I am a surface plot, Hooray :)}
\end{figure}

\subsection{Convergence and mesh refinement}
For \eqref{eq:2a}, the the analytical solution is not available in closed form, 
so in order to analyze convergence we compute a reference solution using a sufficiently high $M$, 
which we use in place of the analytical solution when computing the error. 

In order to analyze the convergence further, 
we now consider the heat equation with a set of boundary and initial conditions for which the analytical solution is known. 
Specifically we consider 
\begin{equation}
    u_t = u_{xx}, \quad u(0,t) = u(1,t) = 0, \quad u(x,0) = \sin(\pi x), 
    \label{eq:2b-manufactured}
\end{equation}
on the same domain $x \in [0,1] := \Omega$ and $t > 0$. 
Note that we now have Dirchlet boundary conditions, 
and the analytical solution is readily available as
\begin{equation}
    u(x,t) = \sin(\pi x)  e^{- \pi^2 t}.
    % Necessary to derive this? Just sepparation of variables
\end{equation}

When doing mesh refinement of the spatial grid, 
we vary the number of spatial grid points $M$, 
and compute the numerical solution at the same point in time $t=t_{end}$. 
The number of time steps $N$ is kept fixed, 
and compute both the $L_2$ discrete relative error. 
For equation \eqref{eq:2b-manufactured} we also compute the $l_2$ continous relative error, 
and the resulting convergence plots are shown in figure \ref{fig:2a-convergence} and \ref{fig:2b-convergence}. 

Both methods are second order in the spatial step $h$, 
but Crank-Nicolson is one order higher in time step $k$. 

\begin{figure}[ht]
    \centering
    \begin{tikzpicture}
    \begin{loglogaxis}[
        title={2a Convergence plot L2 discrete rel. error},
        xlabel={$M$},
        ylabel={Relative error},
        ymax=10,
        ]
        \addplot[color=blue,mark=x] table[x={M}, y={err}] 
            {exercise2/data_ka/2a_BE_discrete_err_N100_Mref1000_tend1.dat};
        \addplot[color=red,mark=*] table[x={M}, y={err}] 
            {exercise2/data_ka/2a_CN_discrete_err_N100_Mref1000_tend1.dat};
        \legend{Backward Euler, Crank-Nicolson};
    \end{loglogaxis}
\end{tikzpicture}

    \label{fig:2a-convergence}
\end{figure}

\begin{figure}[ht]
    \centering
    \begin{tikzpicture}
    \begin{groupplot}
        [
            group style={group size=2 by 1, horizontal sep=2cm},
            height=7cm,
            width=0.48\textwidth,
        ]
        \nextgroupplot[title=2b discrete, xmode=log, ymode=log, legend pos=south west]
        \addplot[color=blue,mark=x] table[x={M}, y={err}] 
            {exercise2/data_ka/2b_UMR_BE_discrete_err_N1000_tend1.dat};
        \addplot[color=red,mark=*] table[x={M}, y={err}] 
            {exercise2/data_ka/2b_UMR_CN_discrete_err_N1000_tend1.dat};
        \addplot[color=black,mark=diamond] table[x={M}, y={err}] 
            {exercise2/data_ka/2b_UMR_BE_discrete_err_N10000_tend1.dat};
        \addplot[color=gray,mark=triangle] table[x={M}, y={err}] 
            {exercise2/data_ka/2b_UMR_CN_discrete_err_N10000_tend1.dat};
        \legend{BE N=1000, CN N=1000, BE N=10000, CN N=10000};

        \nextgroupplot[title=2b continous, xmode=log, ymode=log, legend pos=south west]
        \addplot[color=blue,mark=x] table[x={M}, y={err}] 
            {exercise2/data_ka/2b_UMR_BE_continous_err_N1000_tend1.dat};
        \addplot[color=red,mark=*] table[x={M}, y={err}] 
            {exercise2/data_ka/2b_UMR_CN_continous_err_N1000_tend1.dat};
        \addplot[color=black,mark=diamond] table[x={M}, y={err}] 
            {exercise2/data_ka/2b_UMR_BE_continous_err_N10000_tend1.dat};
        \addplot[color=gray,mark=triangle] table[x={M}, y={err}] 
            {exercise2/data_ka/2b_UMR_CN_continous_err_N10000_tend1.dat};
        \legend{BE N=1000, CN N=1000, BE N=10000, CN N=10000};
    \end{groupplot}
\end{tikzpicture}

    \label{fig:2b-convergence}
\end{figure}

%%%%%%%%%%%%%%%%%%%%%%%%%%%%%%%%%%%%%%%%%%%%%%%%%%%%%%%%%%%%%%%
% TODO: AMR data is likely to change.
% Luckily, that isn't a big problem, as pgfplots is brilliant.

% Seriously. It's glorious.
%%%%%%%%%%%%%%%%%%%%%%%%%%%%%%%%%%%%%%%%%%%%%%%%%%%%%%%%%%%%%%%
%
%        ,,,,,             pgfplots
%       ////""\               .
%      (((/ m m              -|-                        __
%      )))c  = )              |                        (__)
%     ////-./~`    .                                    []
%    (((( `.`\    ::                                    []
%     )))`\ \)).-;.'                           .------, []
%      (() `._.-'`                           _(        )[]
%      )/ `. |  .'`^^^^^^^^^^^^^^^^^^^^^^^^^^))\`.----'`[]
%jgs   (    \' { ~ - ~~ _  ~  -  ~~  - ~  - ((  | |     []
%  .-.--\    \ {                             )) | |     []
%  |_;_._`\   |{                            ((__|_|-----[]
% |  ;   ```  ;{                             ))         []
% | /``-.____/ `~~~[]~~~~~~~~~~~~~~~~~~~~~~~'-'         []
% `'              (__)                                 (__)
%\begin{figure}[ht]
%    \centering
%    \begin{tikzpicture}
    \begin{groupplot}
        [
            group style={group size=2 by 1, horizontal sep=2cm},
            height=7cm,
            width=0.48\textwidth,
        ]
        \nextgroupplot[title=2b AMR discrete, xmode=log, ymode=log, legend pos=south west]
        \addplot[color=blue,mark=x] table[x={M}, y={err}] 
            {exercise2/data_ka/2b_AMR_BE_discrete_err_N10000_tend1.dat};
        \addplot[color=red,mark=*] table[x={M}, y={err}] 
            {exercise2/data_ka/2b_AMR_CN_discrete_err_N10000_tend1.dat};
        \addplot[color=black,mark=diamond] table[x={M}, y={err}] 
            {exercise2/data_ka/2b_UMR_BE_discrete_err_N10000_tend1.dat};
        \addplot[color=gray,mark=triangle] table[x={M}, y={err}] 
            {exercise2/data_ka/2b_UMR_CN_discrete_err_N10000_tend1.dat};
        \legend{BE AMR, CN AMR, BE UMR, CN UMR};

        \nextgroupplot[title=2b AMR continous, xmode=log, ymode=log, legend pos=south west]
        \addplot[color=blue,mark=x] table[x={M}, y={err}] 
            {exercise2/data_ka/2b_AMR_BE_continous_err_N10000_tend1.dat};
        \addplot[color=red,mark=*] table[x={M}, y={err}] 
            {exercise2/data_ka/2b_AMR_CN_continous_err_N10000_tend1.dat};
        \addplot[color=black,mark=diamond] table[x={M}, y={err}] 
            {exercise2/data_ka/2b_UMR_BE_continous_err_N10000_tend1.dat};
        \addplot[color=gray,mark=triangle] table[x={M}, y={err}] 
            {exercise2/data_ka/2b_UMR_CN_continous_err_N10000_tend1.dat};
        \legend{BE AMR, CN AMR, BE UMR, CN UMR};
    \end{groupplot}
\end{tikzpicture}

%\end{figure}

\section{Invicid Burgers' equation}

In this section we briefly inspect and solve the inviscid Burgers' equation, 
and evaluate the breaking which this equation exhibits. 

Notes:
    -Also semi-discretization, but integrate using off-shelf routine
    -Breaking/shock formation: Numerical vs theoretical

\begin{figure}[ht]
    \centering
    \begin{tikzpicture}
    \begin{groupplot}
        [
            group style={group size=2 by 2, horizontal sep=1cm},
            height=6cm,
            width=0.48\textwidth,
        ]
        \nextgroupplot[title=2c, legend pos=south west, ymin=0.8, xmin=0.5, xmax=0.6]
        \addplot[color=blue] table[x={x}, y={0.05400000000000004}] 
            {exercise2/data_ka/2c_sols_M1000_tf0.06_tbreak0.05500000000000004.dat};
        \legend{one timestep before t break}

        \nextgroupplot[title=2c, legend pos=south west, ymin=0.8, xmin=0.5, xmax=0.6]
        \addplot[color=blue] table[x={x}, y={0.05500000000000004}] 
            {exercise2/data_ka/2c_sols_M1000_tf0.06_tbreak0.05500000000000004.dat};
        \legend{t break}

        \nextgroupplot[title=2c, legend pos=south west, ymin=0.8, xmin=0.5, xmax=0.6]
        \addplot[color=blue] table[x={x}, y={0.05600000000000004}] 
            {exercise2/data_ka/2c_sols_M1000_tf0.06_tbreak0.05500000000000004.dat};
        \legend{one timestep after t break}

        \nextgroupplot[title=2c, legend pos=south west, ymin=0.8, xmin=0.5, xmax=0.6]
        \addplot[color=blue] table[x={x}, y={0.06}] 
            {exercise2/data_ka/2c_sols_M1000_tf0.06_tbreak0.05500000000000004.dat};
        \legend{t final, 0.06}

    \end{groupplot}
\end{tikzpicture}

\end{figure}
