\section{Laplace equation in two dimensions}
In this section, we will solve the two-dimensional Laplace equation on a quadratic domain
\begin{equation}
    u_{xx} + u_{yy} = 0, \, (x,y) \in \Omega := [0,1]^2,
    \label{ex3:eq:laplace}
\end{equation} with boundary conditions on the edges of $\Omega$
\begin{equation*}
    \begin{split}
        u(0,y) &= 0,\\
        u(1,y) &= 0,\\
        u(x,0) &= 0,\\
        u(x,1) &= \sin(2\pi x).\\
    \end{split}
\end{equation*}
We will solve this equation numerically using a five point stencil, but first, we solve it analytically to provide a reference solution which can be compared with the numerical one.

\subsection{Analytical solution}

The solution of equation \ref{ex3:eq:laplace} can be found by separation of variables.
First, asssume that we can write
\begin{equation*}
    % I used these names, but I think it's more common with capital X and Y or something.
    % I'm open for discussions of this.
    u(x,y) = \alpha(x) \beta(y),
\end{equation*}
which implies that
\begin{equation*}
    u_{xx} + u_{yy} =  \alpha''(x) \beta(y) + \alpha(x) \beta''(y) = 0,
\end{equation*}
where the prime markers $'$ denote differentiation of the single variable functions $\alpha(x)$ and $\beta(y)$.
Rearranging, we get that
\begin{equation*}
    \frac{\alpha''(x)}{\alpha(x)} = \frac{\beta''(y)}{\beta(y)} = c
\end{equation*}
must be constant, since $\alpha$ and $\beta$ are functions of indepentent variables.
Thus, we have two second order differential equations
\begin{equation*}
    \begin{split}
        \alpha''(x) - c\alpha(x) &= 0, \\
        \beta''(x) - c\beta(x) &= 0,
    \end{split}
\end{equation*}
with boundary conditions
\begin{equation*}
    \begin{split}
    \alpha(0) = \alpha(1) = \beta(0) = 0,\\
    \alpha(x)\beta(1) = \sin(2\pi x).
    \end{split}
\end{equation*}

Setting $\beta(1)$ to $1$ yields $\alpha(x) = \sin(2\pi x)$, so that $\alpha''(x) = -4\pi^2\alpha(x)$ where $y = 1$, we find that $c = -4\pi^2$.
Solving the equation for $\beta(y)$, we find that
\begin{equation*}
    \beta(y) = b_1 e^{\sqrt{c} y} + b_2 e^{-\sqrt{c} y}.
\end{equation*}
Inserting $c = 4\pi$ and the boundary condtitions $\beta(0) = 0$ and $\beta(1) = 1$, we get
\begin{equation*}
    \beta(y) = \frac{\sinh(2\pi y)}{\sinh(2\pi)},
\end{equation*}
and finally
\begin{equation*}
    u(x,y) = \frac{\sin(2\pi x) \cdot \sinh(2\pi y)}{\sinh(2\pi)}.
\end{equation*}

\subsection{Numerical solution}

Numerically, we can solve this equation by dividing the domain $\Omega$ into a grid.
That is, we divide the interval $x \in [0,1]$ and $y \in [0,1]$ into $M$ and $N$ parts, respectively, so that the domain contains $M * N$ points, in which we will approximate the solution to equation \ref{ex3:eq:laplace}.

Rewriting Laplace's equation using central differences, we get
\begin{equation*}
    \begin{split}
    % TODO: What do you people think about this notation?
    % I think the equations are a bit long, but relatively clear in return.
    \partial^2_x u(x_m, y_n) 
        &= \frac{1}{h^2}[u(x_{m-1},y_n) + 2u(x_m,y_n) + u(x_{m+1},y_n)] + \Oh(h^2)\\
        &= \frac{1}{h^2}\delta^2_x u(x_m,y_n),\\
    \end{split}
\end{equation*}
\begin{equation*}
    \begin{split}
    \partial^2_y u(x_m, y_n) 
        &= \frac{1}{k^2}[u(x_{m},y_{n-1}) + 2u(x_m,y_n) + u(x_{m},y_{n+1})] + \Oh(k^2)\\
        &= \frac{1}{h^2}\delta^2_y u(x_m,y_n),\\
    \end{split}
\end{equation*}
where $(x_m, y_n)$ denote the point $(m,n)$ in the grid. 
%$i = 0,\ldots M_x$, and $j = 0,\ldots, M_y$.
Adding these expressions, and naming our approximated solution $U_{m,n} := u(x_m,y_n)$, we find that the Laplace equation can be approximated 
\begin{equation*}
    0 = \partial^2_x u(x_m,y_n) + \partial^2_y u(x_m,y_n)
    \approx \frac{1}{h^2}\delta^2_x U_{m,n} + \frac{1}{k^2}\delta^2_y U_{m,n},
\end{equation*}
or, simplifying the notation with the notation visualized in figure \ref{ex3:fig:stencil},
\begin{equation*}
    \frac{1}{k^2}(U_{above} + U_{below} - 2U_{center}) + \frac{1}{h^2}(U_{left} + U_{right} - 2U_{center}) = 0.
\end{equation*}

\begin{figure}[hb]
    \centering
    \begin{tikzpicture}
    %% X-axis, courtesy of task force 1:
    \draw[dashed] (0,2) -- (4,2);
    \draw[dashed](2,0) -- (2,4);
    \filldraw (0,2) circle (2pt) node[anchor=north] {$U_{left}$};
    \filldraw (4,2) circle (2pt) node[anchor=north] {$U_{right}$};
    \filldraw (2,0) circle (2pt) node[anchor=north] {$U_{below}$};
    \filldraw (2,4) circle (2pt) node[anchor=south] {$U_{above}$};
    \filldraw (2,2) circle (2pt) node[anchor=south west] {$U_{center}$};

\end{tikzpicture}

    \label{ex3:fig:stencil}
    \caption{The five-point stencil corresponding to central difference differentiation in both the $x$- and $y$-direction.} 
\end{figure}

This stencil can be used to approximate the value of $U(x_m, y_n) = U_{center}$ for all points $(x_m, y_n)$ in the grid.
Ignoring firstly the above and below nodes of the stencil, we can easily set up a matrix $A'$ in the same way as in Section \ref{task_1}.
Note that this is done only in order to clarify the derivation -- the matrix $A'$ is merely a "stepping stone" -- not a useful result.
\begin{equation*}
    %\renewcommand{\arraystretch}{2.5} % stretch matrix vertically to make it square
    A'U = \frac{1}{h^2}
    \begin{bmatrix}
    -2& 1 \\
    1 & -2 & 1 &   \\
      & \ddots & \ddots & \ddots & \\
      %&   & 1 & -2 & 1 \\
      &   & 1 & -2 & 1 \\
      &   &  & 1 & -2 \\
    \end{bmatrix}
    \begin{bmatrix}
    U_{0n} \\ U_{1n} \\ \vdots \\ U_{M,n} \\ U_{M+1,n} \\
    \end{bmatrix}
    =
    \begin{bmatrix}
    0 \\ 0 \\ \vdots \\ 0 \\ 0 \\
    \end{bmatrix}
    ,
    \label{ex3:eq:simple_matrix}
\end{equation*}
Note also that this equation only considers one particular $n$.
In order to actually solve our entire system, we must include the nodes above and below as well.
This can be done by considering a much larger matrix $A$ and a much longer vector $U$.
The latter being a stacked vector containing all $M+1$ elements $U_{0n}, \ldots, U_{Mn}$, followed by $U_{01}, \ldots U_{M1}$ and so on.
The matrix $A$ is still the tridiagonal matrix sith elements $[1, -4, 1]$, but now with size $(N\cdot M)^2$.
Now, we are able to inclute the nodes $U_{above}$ and $U_{below}$ from the stencil.
These two nodes are now -- instead of being above and below $U_{center}$ -- to the sides, $M$ nodes away. See figure \ref{ex3:fig:flat_stencil}.

\begin{figure}[htb]
    \centering
    \begin{tikzpicture}
    \draw[dashed] (0.0,1.5) -- (3,1.5);
    \draw[dashed](1.5, 0.0) -- (1.5,3);
    \filldraw (0,1.5) circle (2pt) node[anchor=north] {$U_{m-1,n}$};
    \filldraw (3,1.5) circle (2pt) node[anchor=north] {$U_{m+1,n}$};
    \filldraw (1.5,0) circle (2pt) node[anchor=north] {$U_{m,n+1}$};
    \filldraw (1.5,3) circle (2pt) node[anchor=south] {$U_{m,n-1}$};
    \filldraw (1.5,1.5) circle (2pt) node[anchor=south west] {$U_{m,n}$};

    \draw[->] (4,1.5) -- (6,1.5);

    \draw[dotted] (7, 1.5) -- (9, 1.5);
    \draw[dashed] (9, 1.5) -- (12,1.5);
    \draw[dotted] (12,1.5) -- (14,1.5);
    \filldraw (10.5,1.5) circle (2pt) node[anchor=north] {$U_{m,n}$};
    \filldraw (9.00,1.5) circle (2pt) node[anchor=north] {$U_{m-1,n}$};
    \filldraw (12.0,1.5) circle (2pt) node[anchor=north] {$U_{m+1,n}$};
    \filldraw (7.00,1.5) circle (2pt) node[anchor=north] {$U_{m,n-1}$};
    \filldraw (14.0,1.5) circle (2pt) node[anchor=north] {$U_{m,n+1}$};
\end{tikzpicture}

    \label{ex3:fig:flat_stencil}
    \caption{By flattening the five-point stencil, we can write the system of equations, which is now on the form $AU = 0$.} 
\end{figure}

We thus write
\begin{equation*}
    \renewcommand{\arraystretch}{2.5} % stretch matrix vertically to make it square
    % This matrix doesn't look good. We could divide it, or give the elements names (e.g. a, b, c),
    % or keep things like this. I don't know what's best.
    AU = 
    \begin{bmatrix}
        \frac{-2}{h^2} + \frac{-2}{k^2} & \frac{1}{h^2} &&\frac{1}{k^2}\\
        \frac{1}{h^2} & \frac{-2}{h^2} + \frac{-2}{k^2}   & \frac{1}{h^2} &&\frac{1}{k^2}\\
        & \frac{1}{h^2} & \frac{-2}{h^2} + \frac{-2}{k^2}   & \frac{1}{h^2} &&\frac{1}{k^2}\\
        &  \quad \ddots & \quad \ddots  & \quad \ddots \\
        \frac{1}{k^2} && \frac{1}{h^2} & \frac{-2}{h^2} + \frac{-2}{k^2} & \frac{1}{h^2} \\
        & \frac{1}{k^2} && \frac{1}{h^2} & \frac{-2}{h^2} + \frac{-2}{k^2}   & \frac{1}{h^2} \\
        && \frac{1}{k^2} && \frac{-2}{h^2} + \frac{-2}{k^2} & \frac{1}{h^2} \\
    \end{bmatrix}
    \begin{bmatrix}
    U_{0} \\ \vdots \\ U_{m} \\ \vdots \\ U_{N \cdot m} \\ \vdots \\ U_{N \cdot M} \\
    \end{bmatrix}
    =
    \begin{bmatrix}
    0 \\ 0 \\ 0 \\ \vdots \\ 0 \\ 0 \\ 0 \\
    \end{bmatrix}
    = 0,
    \label{ex3:eq:solution_equation}
\end{equation*}
which can be solved, given initial conditions on the boundary of $\Omega$.

\begin{figure}[htbp]
    \centering
    \begin{tikzpicture}
        \begin{axis}[
            colorbar,
            view={0}{90},
            colormap/jet,
            mesh/ordering=x varies,
            mesh/cols=100,
            mesh/rows=100,
        ]
            \addplot[
                surf,
                shader=flat,
                point meta=explicit,
            ] 
            table[meta=U]{./exercise3/laplace.dat};
        \end{axis}
    \end{tikzpicture}
    \label{ex3:fig:heat_map}
    \caption{The numerically computed solution to the Laplace equation.}
\end{figure}
