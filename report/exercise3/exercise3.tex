\section{Laplace equation in two dimensions}
In this section, we will solve the two-dimensional Laplace equation on a quadratic domain
\begin{equation}
    u_{xx} + u_{yy} = 0, \, (x,y) \in \Omega := [0,1]^2,
    \label{ex3:eq:laplace}
\end{equation} with boundary conditions on the edges of $\Omega$
\begin{equation*}
    \begin{split}
        u(0,y) &= 0,\\
        u(1,y) &= 0,\\
        u(x,0) &= 0,\\
        u(x,1) &= \sin(2\pi x).\\
    \end{split}
\end{equation*}
We will solve this equation numerically using a five point stencil, but first, we solve it analytically to provide a reference solution which can be compared with the numerical one.

The solution of equation \ref{ex3:eq:laplace} can be found by separation of variables.
First, asssume that we can write
\begin{equation*}
    % I used these names, but I think it's more common with capital X and Y or something.
    % I'm open for discussions of this.
    u(x,y) = \alpha(x) \beta(y),
\end{equation*}
which implies that
\begin{equation*}
    u_{xx} + u_{yy} =  \alpha''(x) \beta(y) + \alpha(x) \beta''(y) = 0,
\end{equation*}
where the prime markers $'$ denote differentiation of the single variable functions $\alpha(x)$ and $\beta(y)$.
Rearranging, we get that
\begin{equation*}
    \frac{\alpha''(x)}{\alpha(x)} = \frac{\beta''(y)}{\beta(y)} = c
\end{equation*}
must be constant, since $\alpha$ and $\beta$ are functions of indepentent variables.
Thus, we have two second order differential equations
\begin{equation*}
    \begin{split}
        \alpha''(x) - c\alpha(x) &= 0, \\
        \beta''(x) - c\beta(x) &= 0,
    \end{split}
\end{equation*}
with boundary conditions
\begin{equation*}
    \begin{split}
    \alpha(0) = \alpha(1) = \beta(0) = 0,\\
    \alpha(x)\beta(1) = \sin(2\pi x).
    \end{split}
\end{equation*}

Setting $\beta(1)$ to $1$ yields $\alpha(x) = \sin(2\pi x)$, so that $\alpha''(x) = -4\pi^2\alpha(x)$ where $y = 1$, we find that $c = -4\pi^2$.
Solving the equation for $\beta(y)$, we find that
\begin{equation*}
    \beta(y) = b_1 e^{\sqrt{c} y} + b_2 e^{-\sqrt{c} y}.
\end{equation*}
Inserting $c = 4\pi$ and the boundary condtitions $\beta(0) = 0$ and $\beta(1) = 1$, we get
\begin{equation*}
    \beta(y) = \frac{\sinh(2\pi y)}{\sinh(2\pi)},
\end{equation*}
and finally
\begin{equation*}
    u(x,y) = \frac{\sin(2\pi x) \cdot \sinh(2\pi y)}{\sinh(2\pi)}.
\end{equation*}

Numerically, we can solve this equation by dividing the domain $\Omega$ into a grid.
That is, we divide the interval $x \in [0,1]$ and $y \in [0,1]$ into $M$ and $N$ parts, respectively, so that the domain contains $M * N$ points, in which we will approximate the solution to equation \ref{ex3:eq:laplace}.

Rewriting Laplace's equation using central differences, we get
\begin{equation*}
    \begin{split}
    % This notation is indonsistent with task 1, but it is explicit 
    % regarding in which points we evaluate the function.
    % TODO: What do you people think?
    \partial^2_x u(x_m, y_n) 
        &= \frac{1}{h^2}[u(x_{m-1},y_n) + 2u(x_m,y_n) + u(x_{m+1},y_n)] + O(h^2)\\
        &= \frac{1}{h^2}\delta^2_x u(x_m,y_n),\\
    \end{split}
\end{equation*}
\begin{equation*}
    \begin{split}
    \partial^2_y u(x_m, y_n) 
        &= \frac{1}{k^2}[u(x_{m},y_{n-1}) + 2u(x_m,y_n) + u(x_{m},y_{n+1})] + O(k^2)\\
        &= \frac{1}{h^2}\delta^2_y u(x_m,y_n),\\
    \end{split}
\end{equation*}
where $(x_m, y_n)$ denote the point $(m,n)$ in the grid. 
%$i = 0,\ldots M_x$, and $j = 0,\ldots, M_y$.
Adding these expressions, and naming our approximated solution $U(m,n) := u(x_m,y_n)$, we find that the function $u_{xx} + u_{yy}$ can be approximated 
\begin{equation*}
    0 = u_{xx}(x_m,y_n) + u_{yy}(x_m,y_n)
    \approx \frac{1}{h^2}\delta^2_x U(m,n) + \frac{1}{k^2}\delta^2_y U(m,n),
\end{equation*}
or, simplifying the notation as shown in figure % TODO: \ref{ex3:fig:stencil}
,
\begin{equation*}
    U_{above} + U_{below} + U_{left} + U_{right} - 4_{center} = 0.
\end{equation*}
This stencil can be used to approximate the value of $U(x_m, y_n) = U_{center}$ for all points $(x_m, y_n)$ in the grid.
