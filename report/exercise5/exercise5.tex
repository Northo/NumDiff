\section{Poisson equation in one dimension finite element method}

\newcommand\basisexpand[2]{\sum_{i=0}^{M+1} #1_i \varphi_i(#2)}
\newcommand\integral[4]{\int_#3^#4 \dif #2 \, #1}

In this section, we will again solve the Poisson equation
\begin{equation}
	-\pd[2]{u}{x} = f(x), \quad u(a) = \alpha, \quad u(b) = \beta, \quad (a \leq x \leq b)
	\label{poisson_equation2}
\end{equation}
subject to Dirichlet conditions, but this time with finite elements instead of finite differences.

\subsection{Analytical solution}

Same as in \cref{poisson_analytical_solution}, but with $f(x) \rightarrow -f(x)$.

\subsection{Numerical solution using finite element method}

Write
\begin{equation}
	u(x) = \hat{u}(x) + r(x), \quad \text{with} \quad \hat{u}(a) = \hat{u}(b) = 0 \quad \text{and} \quad \quad r(x) = \alpha \frac{x-b}{a-b} + \beta \frac{x-a}{b-a}
\end{equation}
Now multiply \cref{poisson_equation2} by a function $v(x)$, integrate both sides from $a$ to $b$ and use integration by parts on the left and $\hat{u}(a) = \hat{u}(b) = 0$ to drop the boundary term.
This gives the \textbf{weak formulation} of the problem: find $\hat{u}$ such that for all $v$
\newcommand{\weakform}[3]{\integral{\hat{#1}'(x) #2'(x)}{x}{a}{b} = \integral{f(x) #2(x)}{x}{a}{b} - \integral{#3'(x) #2'(x)}{x}{a}{b}}
\begin{equation}
	\weakform{u}{v}{r}
	% \integral{\hat{u}'(x) v'(x)}{x}{a}{b} = \integral{f(x) v(x)}{x}{a}{b} - \integral{R'(x) v'(x)}{x}{a}{b}
	\label{weak_form_exact}
\end{equation}
\Cref{weak_form_exact} is fully equivalent to \cref{poisson_equation2}.

The approximation lies in seeking a solution $U(x) = \hat{U}(x) + R(x) \approx u(x)$ belonging to the space of piecewise linear functions.
Divide the interval $[a, b]$ into the finite elements
\begin{equation*}
	a = x_0 < x_1 < \dots < x_M < x_{M+1} = b.
\end{equation*}
and let $U(x)$ be piecewise linear between these points.
This leads to the weak formulation
\begin{equation}
	\weakform{U}{V}{R}
	% \integral{\hat{u}'(x) v'(x)}{x}{a}{b} = \integral{f(x) v(x)}{x}{a}{b} - \integral{R'(x) v'(x)}{x}{a}{b}
	\label{weak_form_approximate}
\end{equation}
Expand $U(x) = \basisexpand{U}{x}$ and $V(x) = \basisexpand{V}{x}$ in
\begin{equation*}
	\varphi_i(x) = 
	\begin{cases}
		(x - x_{i-1})/(x_i - x_{i-1}) & \text{if}        \,\, x_{i-1} \leq x \leq x_i     \\
		(x_{i+1} - x)/(x_{i+1} - x_i) & \text{if}        \,\, x_i     \leq x \leq x_{i+1} \\
		0                             & \text{otherwise} \\
	\end{cases}
	=
	%\begin{tikzpicture}
	\begin{tikzpicture}[baseline={([yshift=-.5ex]current bounding box.center)}]
	\begin{axis}[width=6cm,height=4cm,xmin=-3, xmax=+3]
	\addplot [color=red, domain=-4:-2] {0};
	\addplot [color=red, domain=-2: 0] {(x - -2) / (0 - -2)};
	\addplot [color=red, domain= 0:+1] {(1 -  x) / (1 -  0)};
	\addplot [color=red, domain=+1:+3] {0};
	\node (0.5, 0.5) {hey};
	\end{axis}
	\end{tikzpicture}
	\label{basisfunction}
\end{equation*}
and insert back to get
\begin{equation*}
\begin{split}
	\sum_{i, j} \hat{U}_i V_j \integral{\varphi'_i(x) \varphi'_j(x)}{x}{a}{b}
	&= \sum_j V_j \integral{\varphi_j(x) f(x)}{x}{a}{b} \\
	&- a \sum_j V_j \integral{\varphi'_0(x) \varphi'_j(x)}{x}{a}{b}
	- b \sum_j V_j \integral{\varphi'_{M+1}(x) \varphi'_j(x)}{x}{a}{b} \\
\end{split}
\end{equation*}
or $V^T A \hat{U} = V^T F$ where then
\begin{equation}
	A \hat{U} = F
	\quad \text{with} \quad
	A_{ij} = \integral{\varphi'_i(x) \varphi'_j(x)}{x}{a}{b} 
	\quad \text{and} \quad
	F_j = \integral{\varphi_j(x)}{x}{a}{b}
\end{equation}
The A integrals are straightforward to differentiate \ref{basisfunction}:
\begin{equation}
\begin{split}
	A_{00} = \frac{1}{x_1 - x_0} \\
	A_{ii} = \frac{1}{x_i - x_{i-1}} + \frac{1}{x_{i+1}-x_i} \\
	A_{M+1 \, M+1} = \frac{1}{x_{M+1}-x_M} \\
	A_{i i+1} = A_{i+1 i} = \frac{-1}{x_{i+1} - x_i}
\end{split}
\end{equation}

We now impose $\hat{U}(a) = \hat{U}(b) = 0$ by removing the first and last entries in the matrix equation \emph{after} calculating the right side.
This gives a $M \times M$ equation.
Then we reconstruct the full solution by adding $\alpha$ and $\beta$ as the first and last entries to get $M+2$ points.
From $U$, we reconstruct $U(x) = \basisexpand{U}{x}$.

\newcommand\plotseries[5]{
\pgfplotstablegetrowsof{#1-step0.dat}
\pgfmathsetmacro{\MA}{\pgfplotsretval}
\pgfplotstablegetrowsof{#1-step1.dat}
\pgfmathsetmacro{\MB}{\pgfplotsretval}
\pgfplotstablegetrowsof{#1-step2.dat}
\pgfmathsetmacro{\MC}{\pgfplotsretval}

\begin{tikzpicture}
\begin{groupplot}[
	group style={group size=3 by 2, horizontal sep=0.1cm, vertical sep=0.1cm},
	width=6.4cm,
	xtick={#4, #5}, minor xtick=data,
]

\nextgroupplot[height=3cm, xticklabels={,,}, xtick={#4, #5}, minor xtick=data, ymin=0, ymax=1.2, ytick={0, 1}, minor ytick={0.5}, title={$M=\MA$}, ylabel={$E/E_{\text{max}}$}];
\addplot [ybar interval, fill=red] table [x=x, y=E] {#1-step0.dat}; % errors
\addplot [black, dashed] table [y=refE] {#1-step0.dat}; % reference error line
\nextgroupplot[height=3cm, xticklabels={,,}, xtick={#4, #5}, minor xtick=data, ymin=0, ymax=1.2, ytick={0, 1}, minor ytick={0.5}, title={$M=\MB$}, yticklabels={,,}];
\addplot [ybar interval, fill=red] table [x=x, y=E] {#1-step1.dat}; % errors
\addplot [black, dashed] table [y=refE] {#1-step1.dat}; % reference error line
\nextgroupplot[height=3cm, xticklabels={,,}, xtick={#4, #5}, minor xtick=data, ymin=0, ymax=1.2, ytick={0, 1}, minor ytick={0.5}, title={$M=\MC$}, yticklabels={,,}];
\addplot [ybar interval, fill=red] table [x=x, y=E] {#1-step2.dat}; % errors
\addplot [black, dashed] table [y=refE] {#1-step2.dat}; % reference error line

\nextgroupplot[xlabel=$x$];
\addplot [draw=none] table [x=x, y=U] {#1-step0.dat}; % hidden dummy plot (just for ticks)
\addplot [blue, domain=#4:#5, samples=300, line width=1pt, black] {#2}; % analytical
\addplot [black, mark=*, mark size=1.0, red, line width=0.5pt] table [x=x, y=U] {#1-step0.dat}; % numerical

\nextgroupplot[xlabel=$x$, yticklabels={,,}];
\addplot [draw=none] table [x=x, y=U] {#1-step1.dat}; % hidden dummy plot (just for ticks)
\addplot [blue, domain=#4:#5, samples=300, line width=1pt, black] {#2}; % analytical
\addplot [black, mark=*, mark size=1.0, red, line width=0.5pt] table [x=x, y=U] {#1-step1.dat}; % numerical

\nextgroupplot[xlabel=$x$, yticklabels={,,}];
\addplot [draw=none] table [x=x, y=U] {#1-step2.dat}; % hidden dummy plot (just for ticks)
\addplot [blue, domain=#4:#5, samples=300, line width=1pt, black] {#2}; % analytical
\addplot [black, mark=*, mark size=1.0, red, line width=0.5pt] table [x=x, y=U] {#1-step2.dat}; % numerical
\end{groupplot}
\end{tikzpicture}
}

\newcommand\figureseries[2]{
\begin{figure}
\centering
$f(x) = -2$
\medskip \medskip
\plotseries{#1/f1}{x^2}{}{0}{1} \\
\medskip \medskip \medskip \medskip
$f(x) = (40000x^2 - 200) \exp(-100x^2)$
\medskip \medskip 
\plotseries{#1/f2}{exp(-100) - exp(-100*x^2)}{}{-1}{1} \\
\end{figure}
\begin{figure}
\ContinuedFloat % split over two pages
\centering
$f(x) = (4000000x^2 - 2000) \exp(-1000x^2)$
\medskip \medskip
\plotseries{#1/f3}{exp(-1000) - exp(-1000*x^2)}{}{-1}{1} \\
\medskip \medskip \medskip \medskip
$f(x) = 2x^{-4/3}/9$
\medskip \medskip
\plotseries{#1/f4}{x^(2/3)}{}{0}{1}
\caption{#2}
\end{figure}
}

\figureseries{exercise5/data/UMR}{Uniform refinement}

\figureseries{exercise5/data/AMR}{Adaptive refinement, average strategy}

\begin{figure}
\centering
\begin{tikzpicture}
\begin{groupplot}[
	group style={group size=2 by 2, horizontal sep=1.5cm, vertical sep=2.5cm},
	xmode=log, ymode=log,
	width=8cm, height=7cm, 
	legend entries={uniform, adaptive (avg.), adaptive (max)},
	log basis x=2, xticklabel=\pgfmathparse{2^\tick}\pgfmathprintnumber\pgfmathresult,
]
\pgfplotsinvokeforeach {1,2,3,4} {
	\ifthenelse{\equal{#1}{1}}{
		\nextgroupplot[ylabel=error,title={$f(x) = -2$}];
	}{\ifthenelse{\equal{#1}{2}}{
		\nextgroupplot[title={$f(x) = (40000x^2 - 200) \exp(-100x^2)$}];
	}{\ifthenelse{\equal{#1}{3}}{
		\nextgroupplot[xlabel=$M+1$, ylabel=error, title={$f(x) = (4000000x^2 - 2000) \exp(-1000x^2)$}];
	}{
		\nextgroupplot[xlabel=$M+1$, title={$f(x) = 2x^{-4/3}/9$}];
	}}}
	\addplot table [x=M1, y=E1] {exercise5/data/convergence/f#1-convergence.dat};
	\addplot table [x=M2, y=E2] {exercise5/data/convergence/f#1-convergence.dat};
	\addplot table [x=M3, y=E3] {exercise5/data/convergence/f#1-convergence.dat};
}

\end{groupplot}
\end{tikzpicture}
\end{figure}
